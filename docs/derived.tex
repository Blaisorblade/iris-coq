\section{Derived constructions}

\subsection{Non-atomic (``thread-local'') invariants}

Sometimes it is necessary to maintain invariants that we need to open non-atomically.
Clearly, for this mechanism to be sound we need something that prevents us from opening the same invariant twice, something like the masks that avoid reentrancy on the ``normal'', atomic invariants.
The idea is to use tokens\footnote{Very much like the tokens that are used to encode ``normal'', atomic invariants} that guard access to non-atomic invariants.
Having the token $\NaTokE\pid\mask$ indicates that we can open all invariants in $\mask$.
The $\pid$ here is the name of the \emph{invariant pool}.
This mechanism allows us to have multiple, independent pools of invariants that all have their own namespaces.

One way to think about non-atomic invariants is as ``thread-local invariants'',
where every pool is a thread.
Every thread thus has its own, independent set of invariants.
Every thread threads through all the tokens for its own pool, so that each invariant can only be opened in the thread it belongs to.
As a consequence, they can be kept open around any sequence of expressions (\ie there is no restriction to atomic expressions) -- after all, there cannot be any races with other threads.

Concretely, this is the monoid structure we need:
\begin{align*}
\textdom{PId} \eqdef{}& \GName \\
\textmon{NaTok} \eqdef{}& \finpset{\InvName} \times \pset{\InvName}
\end{align*}
For every pool, there is a set of tokens designating which invariants are \emph{enabled} (closed).
This corresponds to the mask of ``normal'' invariants.
We re-use the structure given by namespaces for non-atomic invariants.
Furthermore, there is a \emph{finite} set of invariants that is \emph{disabled} (open).

Owning tokens is defined as follows:
\begin{align*}
\NaTokE\pid\mask \eqdef{}& \ownGhost{\pid}{ (\emptyset, \mask) } \\
\NaTok\pid \eqdef{}& \NaTokE\pid\top
\end{align*}

Next, we define non-atomic invariants.
To simplify this construction,we piggy-back into ``normal'' invariants.
\begin{align*}
  \NaInv\pid\namesp\prop \eqdef{}& \Exists \iname\in\namesp. \knowInv\namesp{\prop * \ownGhost\pid{(\set{\iname},\emptyset)} \lor \NaTokE\pid{\set{\iname}}}
\end{align*}


We easily obtain:
\begin{mathpar}
  \axiom
  {\TRUE \vs[\bot] \Exists\pid. \NaTok\pid}

  \axiom
  {\NaTokE\pid{\mask_1 \uplus \mask_2} \Lra \NaTokE\pid{\mask_1} * \NaTokE\pid{\mask_2}}
  
  \axiom
  {\later\prop  \vs[\namesp] \always\NaInv\pid\namesp\prop}

  \axiom
  {\NaInv\pid\namesp\prop \proves \Acc[\namesp]{\NaTokE\pid\namesp}{\later\prop}}
\end{mathpar}
from which we can derive
\begin{mathpar}
  \infer
  {\namesp \subseteq \mask}
  {\NaInv\pid\namesp\prop \proves \Acc[\namesp]{\NaTokE\pid\mask}{\later\prop * \NaTokE\pid{\mask \setminus \namesp}}}
\end{mathpar}

\subsection{Boxes}

The idea behind the \emph{boxes} is to have an assertion $\prop$ that is actually split into a number of pieces, each of which can be taken out and back in separately.
In some sense, this is a replacement for having an ``authoritative PCM of Iris assertions itself''.
It is similar to the pattern involving saved propositions that was used for the barrier~\cite{iris2}, but more complicated because there are some operations that we want to perform without a later.

Roughly, the idea is that a \emph{box} is a container for an assertion $\prop$.
A box consists of a bunch of \emph{slices} which decompose $\prop$ into a separating conjunction of the assertions $\propB_\sname$ governed by the individual slices.
Each slice is either \emph{full} (it right now contains $\propB_\sname$), or \emph{empty} (it does not contain anything currently).
The assertion governing the box keeps track of the state of all the slices that make up the box.
The crux is that opening and closing of a slice can be done even if we only have ownership of the boxes ``later'' ($\later$).

The interface for boxes is as follows:
The two core assertions are: $\BoxSlice\namesp\prop\sname$, saying that there is a slice in namespace $\namesp$ with name $\sname$  and content $\prop$; and $\ABox\namesp\prop{f}$, saying that $f$ describes the slices of a box in namespace $\namesp$, such that all the slices together contain $\prop$. Here, $f$ is of type $\nat \fpfn \BoxState$ mapping names to states, where $\BoxState \eqdef \set{\BoxFull, \BoxEmp}$.
\begin{mathpar}
  \inferH{Box-create}{}
  {\TRUE \vs[\namesp] \ABox\namesp\TRUE\emptyset}

  \inferH{Slice-insert-empty}{}
  {\lateropt b\ABox\namesp\prop{f} \vs[\namesp] \Exists\sname \notin \dom(f). \always\BoxSlice\namesp\propB\sname * \lateropt b\ABox\namesp{\prop * \propB}{\mapinsert\sname\BoxEmp{f}}}

  \inferH{Slice-delete-empty}
  {f(\sname) = \BoxEmp}
  {\BoxSlice\namesp\propB\sname \proves \lateropt b\ABox\namesp\prop{f} \vs[\namesp] \Exists \prop'. \lateropt b(\later(\prop = \prop' * \propB) * \ABox\namesp{\prop'}{\mapinsert\gname\bot{f}})}

  \inferH{Slice-fill}
  {f(\sname) = \BoxEmp}
  {\BoxSlice\namesp\propB\sname \proves \lateropt b\propB * \later\ABox\namesp\prop{f} \vs[\namesp] \lateropt b\ABox\namesp\prop{\mapinsert\sname\BoxFull{f}}}

  \inferH{Slice-empty}
  {f(\sname) = \BoxFull}
  {\BoxSlice\namesp\propB\sname \proves \lateropt b\ABox\namesp\prop{f} \vs[\namesp] \later\propB * \lateropt b\ABox\namesp\prop{\mapinsert\sname\BoxEmp{f}}}

  \inferH{Box-fill}
  {\All\sname\in\dom(f). f(\sname) = \BoxEmp}
  {\later\prop * \ABox\namesp\prop{f} \vs[\namesp] \ABox\namesp\prop{\mapinsertComp\sname\BoxFull{\sname\in\dom(f)}{f}}}

  \inferH{Box-empty}
  {\All\sname\in\dom(f). f(\sname) = \BoxFull}
  {\ABox\namesp\prop{f} \vs[\namesp] \later\prop * \ABox\namesp\prop{\mapinsertComp\sname\BoxEmp{\sname\in\dom(f)}{f}}}
\end{mathpar}
Above, $\lateropt b \prop$ is syntactic sugar for $\later\prop$ (if $b$ is $1$) or $\prop$ (if $b$ is $0$).
This is essentially an \emph{optional later}, indicating that the lemmas can be applied with \textlog{Box} being owned now or later, and that ownership is returned the same way.

\begingroup
\paragraph{Model.}
\newcommand\BoxM{\textmon{Box}}
\newcommand\SliceInv{\textlog{SliceInv}}

The above rules are validated by the following model.
We need a CMRA as follows:
\begin{align*}
  \BoxState \eqdef{}& \BoxFull + \BoxEmp \\
  \BoxM \eqdef{}& \authm(\maybe{\exm(\BoxState)}) \times \maybe{\agm(\latert \iProp)}
\end{align*}

Now we can define the assertions:
\begin{align*}
  \SliceInv(\sname, \prop) \eqdef{}& \Exists b. \ownGhost\sname{(\authfull b, \munit)} * (b = 1 \Ra \prop) \\
  \BoxSlice\namesp\prop\sname \eqdef{}& \ownGhost\sname{(\munit, \prop)} * \knowInv\namesp{\SliceInv(\sname,\prop)} \\
  \ABox\namesp\prop{f} \eqdef{}& \Exists \propB : \nat \to \Prop. \later\left( \prop = \Sep_{\sname \in \dom(f)} \propB(\sname) \right ) * {}\\
  & \Sep_{\sname \in \dom(f)} \Exists \gname. \ownGhost\sname{(\authfrag f(\sname), \propB(\sname))} * \knowInv\namesp{\SliceInv(\sname,\propB(\sname))}
\end{align*}
\endgroup % Model paragraph

\paragraph{Derived rules.}
Here are some derived rules:
\begin{mathpar}
  \inferH{Slice-insert-full}{}
  {\later\propB * \lateropt b\ABox\namesp\prop{f} \vs[\namesp] \Exists\sname \notin \dom(f). \always\BoxSlice\namesp\propB\sname * \lateropt b\ABox\namesp{\prop * \propB}{\mapinsert\sname\BoxFull{f}}}

  \inferH{Slice-insert-empty}
  {f(\sname) = \BoxFull}
  {\BoxSlice\namesp\propB\sname \proves \lateropt b \ABox\namesp\prop{f} \vs[\namesp] \later\propB * \Exists \prop'. \lateropt b (\later(\prop = \prop' * \propB) * \ABox\namesp{\prop'}{\mapinsert\sname\bot{f}})}

  \inferH{Slice-split}
  {f(\sname) = s}
  {\kern-4ex\BoxSlice\namesp{\propB_1 * \propB_2}\sname \proves \lateropt b \ABox\namesp\prop{f} \vs[\namesp] \Exists \sname_1 \notin \dom(f), \sname_2 \notin \dom(f). \sname_1 \neq \sname_1 \land {}\\\kern5ex \always\BoxSlice\namesp{\propB_1}{\sname_1} * \always\BoxSlice\namesp{\propB_2}{\sname_2} * \lateropt b \ABox\namesp\prop{\mapinsert{\sname_2}{s}{\mapinsert{\sname_1}{s}{\mapinsert\sname\bot{f}}}}}

  \inferH{Slice-merge}
  {\sname_1 \neq \sname_2 \and f(\sname_1) = f(\sname_2) = s}
  {\BoxSlice\namesp{\propB_1}{\sname_1}, \BoxSlice\namesp{\propB_2}{\sname_2} \proves \lateropt b \ABox\namesp\prop{f} \vs[\namesp] \Exists \sname \notin \dom(f) \setminus \set{\sname_1, \sname_2}. {}\\\kern5ex \always\BoxSlice\namesp{\propB_1 * \propB_2}\sname * \lateropt b \ABox\namesp\prop{\mapinsert\sname{s}{\mapinsert{\sname_2}{\bot}{\mapinsert{\sname_1}{\bot}{f}}}}}
\end{mathpar}




% TODO: These need syncing with Coq
% \subsection{STSs with interpretation}\label{sec:stsinterp}

% Building on \Sref{sec:stsmon}, after constructing the monoid $\STSMon{\STSS}$ for a particular STS, we can use an invariant to tie an interpretation, $\pred : \STSS \to \Prop$, to the STS's current state, recovering CaReSL-style reasoning~\cite{caresl}.

% An STS invariant asserts authoritative ownership of an STS's current state and that state's interpretation:
% \begin{align*}
%   \STSInv(\STSS, \pred, \gname) \eqdef{}& \Exists s \in \STSS. \ownGhost{\gname}{(s, \STSS, \emptyset):\STSMon{\STSS}} * \pred(s) \\
%   \STS(\STSS, \pred, \gname, \iname) \eqdef{}& \knowInv{\iname}{\STSInv(\STSS, \pred, \gname)}
% \end{align*}

% We can specialize \ruleref{NewInv}, \ruleref{InvOpen}, and \ruleref{InvClose} to STS invariants:
% \begin{mathpar}
%  \inferH{NewSts}
%   {\infinite(\mask)}
%   {\later\pred(s) \vs[\mask] \Exists \iname \in \mask, \gname.   \STS(\STSS, \pred, \gname, \iname) * \ownGhost{\gname}{(s, \STST \setminus \STSL(s)) : \STSMon{\STSS}}}
%  \and
%  \axiomH{StsOpen}
%   {  \STS(\STSS, \pred, \gname, \iname) \vdash \ownGhost{\gname}{(s_0, T) : \STSMon{\STSS}} \vsE[\{\iname\}][\emptyset] \Exists s\in \upclose(\{s_0\}, T). \later\pred(s) * \ownGhost{\gname}{(s, \upclose(\{s_0\}, T), T):\STSMon{\STSS}}}
%  \and
%  \axiomH{StsClose}
%   {  \STS(\STSS, \pred, \gname, \iname), (s, T) \ststrans (s', T')  \proves \later\pred(s') * \ownGhost{\gname}{(s, S, T):\STSMon{\STSS}} \vs[\emptyset][\{\iname\}] \ownGhost{\gname}{(s', T') : \STSMon{\STSS}} }
% \end{mathpar}
% \begin{proof}
% \ruleref{NewSts} uses \ruleref{NewGhost} to allocate $\ownGhost{\gname}{(s, \upclose(s, T), T) : \STSMon{\STSS}}$ where $T \eqdef \STST \setminus \STSL(s)$, and \ruleref{NewInv}.

% \ruleref{StsOpen} just uses \ruleref{InvOpen} and \ruleref{InvClose} on $\iname$, and the monoid equality $(s, \upclose(\{s_0\}, T), T) = (s, \STSS, \emptyset) \mtimes (\munit, \upclose(\{s_0\}, T), T)$.

% \ruleref{StsClose} applies \ruleref{StsStep} and \ruleref{InvClose}.
% \end{proof}

% Using these view shifts, we can prove STS variants of the invariant rules \ruleref{Inv} and \ruleref{VSInv}~(compare the former to CaReSL's island update rule~\cite{caresl}):
% \begin{mathpar}
%  \inferH{Sts}
%   {\All s \in \upclose(\{s_0\}, T). \hoare{\later\pred(s) * P}{\expr}{\Ret \val. \Exists s', T'. (s, T) \ststrans (s', T') * \later\pred(s') * Q}[\mask]
%    \and \physatomic{\expr}}
%   {  \STS(\STSS, \pred, \gname, \iname) \vdash \hoare{\ownGhost{\gname}{(s_0, T):\STSMon{\STSS}} * P}{\expr}{\Ret \val. \Exists s', T'. \ownGhost{\gname}{(s', T'):\STSMon{\STSS}} * Q}[\mask \uplus \{\iname\}]}
%  \and
%  \inferH{VSSts}
%   {\forall s \in \upclose(\{s_0\}, T).\; \later\pred(s) * P \vs[\mask_1][\mask_2] \exists s', T'.\; (s, T) \ststrans (s', T') * \later\pred(s') * Q}
%   {  \STS(\STSS, \pred, \gname, \iname) \vdash \ownGhost{\gname}{(s_0, T):\STSMon{\STSS}} * P \vs[\mask_1 \uplus \{\iname\}][\mask_2 \uplus \{\iname\}] \Exists s', T'. \ownGhost{\gname}{(s', T'):\STSMon{\STSS}} * Q}
% \end{mathpar}

% \begin{proof}[Proof of \ruleref{Sts}]\label{pf:sts}
%  We have to show
%  \[\hoare{\ownGhost{\gname}{(s_0, T):\STSMon{\STSS}} * P}{\expr}{\Ret \val. \Exists s', T'. \ownGhost{\gname}{(s', T'):\STSMon{\STSS}} * Q}[\mask \uplus \{\iname\}]\]
%  where $\val$, $s'$, $T'$ are free in $Q$.
 
%  First, by \ruleref{ACsq} with \ruleref{StsOpen} and \ruleref{StsClose} (after moving $(s, T) \ststrans (s', T')$ into the view shift using \ruleref{VSBoxOut}), it suffices to show
%  \[\hoareV{\Exists s\in \upclose(\{s_0\}, T). \later\pred(s) * \ownGhost{\gname}{(s, \upclose(\{s_0\}, T), T)} * P}{\expr}{\Ret \val. \Exists s, T, S, s', T'. (s, T) \ststrans (s', T') * \later\pred(s') * \ownGhost{\gname}{(s, S, T):\STSMon{\STSS}} * Q(\val, s', T')}[\mask]\]

%  Now, use \ruleref{Exist} to move the $s$ from the precondition into the context and use \ruleref{Csq} to (i)~fix the $s$ and $T$ in the postcondition to be the same as in the precondition, and (ii)~fix $S \eqdef \upclose(\{s_0\}, T)$.
%  It remains to show:
%  \[\hoareV{s\in \upclose(\{s_0\}, T) * \later\pred(s) * \ownGhost{\gname}{(s, \upclose(\{s_0\}, T), T)} * P}{\expr}{\Ret \val. \Exists s', T'. (s, T) \ststrans (s', T') * \later\pred(s') * \ownGhost{\gname}{(s, \upclose(\{s_0\}, T), T)} * Q(\val, s', T')}[\mask]\]
 
%  Finally, use \ruleref{BoxOut} to move $s\in \upclose(\{s_0\}, T)$ into the context, and \ruleref{Frame} on $\ownGhost{\gname}{(s, \upclose(\{s_0\}, T), T)}$:
%  \[s\in \upclose(\{s_0\}, T) \vdash \hoare{\later\pred(s) * P}{\expr}{\Ret \val. \Exists s', T'. (s, T) \ststrans (s', T') * \later\pred(s') * Q(\val, s', T')}[\mask]\]
 
%  This holds by our premise.
% \end{proof}

% % \begin{proof}[Proof of \ruleref{VSSts}]
% % This is similar to above, so we only give the proof in short notation:

% % \hproof{%
% % 	Context: $\knowInv\iname{\STSInv(\STSS, \pred, \gname)}$ \\
% % 	\pline[\mask_1 \uplus \{\iname\}]{
% % 		\ownGhost\gname{(s_0, T)} * P
% % 	} \\
% % 	\pline[\mask_1]{%
% % 		\Exists s. \later\pred(s) * \ownGhost\gname{(s, S, T)} * P
% % 	} \qquad by \ruleref{StsOpen} \\
% % 	Context: $s \in S \eqdef \upclose(\{s_0\}, T)$ \\
% % 	\pline[\mask_2]{%
% % 		 \Exists s', T'. \later\pred(s') * Q(s', T') * \ownGhost\gname{(s, S, T)}
% % 	} \qquad by premiss \\
% % 	Context: $(s, T) \ststrans (s', T')$ \\
% % 	\pline[\mask_2 \uplus \{\iname\}]{
% % 		\ownGhost\gname{(s', T')} * Q(s', T')
% % 	} \qquad by \ruleref{StsClose}
% % }
% % \end{proof}

% \subsection{Authoritative monoids with interpretation}\label{sec:authinterp}

% Building on \Sref{sec:auth}, after constructing the monoid $\auth{M}$ for a cancellative monoid $M$, we can tie an interpretation, $\pred : \mcarp{M} \to \Prop$, to the authoritative element of $M$, recovering reasoning that is close to the sharing rule in~\cite{krishnaswami+:icfp12}.

% Let $\pred_\bot$ be the extension of $\pred$ to $\mcar{M}$ with $\pred_\bot(\mzero) = \FALSE$.
% Now define
% \begin{align*}
%   \AuthInv(M, \pred, \gname) \eqdef{}& \exists \melt \in \mcar{M}.\; \ownGhost{\gname}{\authfull \melt:\auth{M}} * \pred_\bot(\melt) \\
%   \Auth(M, \pred, \gname, \iname) \eqdef{}& M~\textlog{cancellative} \land \knowInv{\iname}{\AuthInv(M, \pred, \gname)}
% \end{align*}

% The frame-preserving updates for $\auth{M}$ gives rise to the following view shifts:
% \begin{mathpar}
%  \inferH{NewAuth}
%   {\infinite(\mask) \and M~\textlog{cancellative}}
%   {\later\pred_\bot(a) \vs[\mask] \exists \iname \in \mask, \gname.\; \Auth(M, \pred, \gname, \iname) * \ownGhost{\gname}{\authfrag a : \auth{M}}}
%  \and
%  \axiomH{AuthOpen}
%   {\Auth(M, \pred, \gname, \iname) \vdash \ownGhost{\gname}{\authfrag \melt : \auth{M}} \vsE[\{\iname\}][\emptyset] \exists \melt_\f.\; \later\pred_\bot(\melt \mtimes \melt_\f) * \ownGhost{\gname}{\authfull \melt \mtimes \melt_\f, \authfrag a:\auth{M}}}
%  \and
%  \axiomH{AuthClose}
%   {\Auth(M, \pred, \gname, \iname) \vdash \later\pred_\bot(\meltB \mtimes \melt_\f) * \ownGhost{\gname}{\authfull a \mtimes \melt_\f, \authfrag a:\auth{M}} \vs[\emptyset][\{\iname\}] \ownGhost{\gname}{\authfrag \meltB : \auth{M}} }
% \end{mathpar}

% These view shifts in turn can be used to prove variants of the invariant rules:
% \begin{mathpar}
%  \inferH{Auth}
%   {\forall \melt_\f.\; \hoare{\later\pred_\bot(a \mtimes \melt_\f) * P}{\expr}{\Ret\val. \exists \meltB.\; \later\pred_\bot(\meltB\mtimes \melt_\f) * Q}[\mask]
%    \and \physatomic{\expr}}
%   {\Auth(M, \pred, \gname, \iname) \vdash \hoare{\ownGhost{\gname}{\authfrag a:\auth{M}} * P}{\expr}{\Ret\val. \exists \meltB.\; \ownGhost{\gname}{\authfrag \meltB:\auth{M}} * Q}[\mask \uplus \{\iname\}]}
%  \and
%  \inferH{VSAuth}
%   {\forall \melt_\f.\; \later\pred_\bot(a \mtimes \melt_\f) * P \vs[\mask_1][\mask_2] \exists \meltB.\; \later\pred_\bot(\meltB \mtimes \melt_\f) * Q(\meltB)}
%   {\Auth(M, \pred, \gname, \iname) \vdash
%    \ownGhost{\gname}{\authfrag a:\auth{M}} * P \vs[\mask_1 \uplus \{\iname\}][\mask_2 \uplus \{\iname\}]
%    \exists \meltB.\; \ownGhost{\gname}{\authfrag \meltB:\auth{M}} * Q(\meltB)}
% \end{mathpar}


% \subsection{Ghost heap}
% \label{sec:ghostheap}%
% FIXME use the finmap provided by the global ghost ownership, instead of adding our own
% We define a simple ghost heap with fractional permissions.
% Some modules require a few ghost names per module instance to properly manage ghost state, but would like to expose to clients a single logical name (avoiding clutter).
% In such cases we use these ghost heaps.

% We seek to implement the following interface:
% \newcommand{\GRefspecmaps}{\textsf{GMapsTo}}%
% \begin{align*}
%  \exists& {\fgmapsto[]} : \textsort{Val} \times \mathbb{Q}_{>} \times \textsort{Val} \ra \textsort{Prop}.\;\\
%   & \All x, q, v. x \fgmapsto[q] v \Ra x \fgmapsto[q] v \land q \in (0, 1] \\
%   &\forall x, q_1, q_2, v, w.\; x \fgmapsto[q_1] v * x \fgmapsto[q_2] w \Leftrightarrow x \fgmapsto[q_1 + q_2] v * v = w\\
%   & \forall v.\; \TRUE \vs[\emptyset] \exists x.\; x \fgmapsto[1] v \\
%   & \forall x, v, w.\; x \fgmapsto[1] v \vs[\emptyset] x \fgmapsto[1] w
% \end{align*}
% We write $x \fgmapsto v$ for $\exists q.\; x \fgmapsto[q] v$ and $x \gmapsto v$ for $x \fgmapsto[1] v$.
% Note that $x \fgmapsto v$ is duplicable but cannot be boxed (as it depends on resources); \ie we have $x \fgmapsto v \Lra x \fgmapsto v * x \fgmapsto v$ but not $x \fgmapsto v \Ra \always x \fgmapsto v$.

% To implement this interface, allocate an instance $\gname_G$ of $\FHeap(\Val)$ and define
% \[
% 	x \fgmapsto[q] v \eqdef
% 	  \begin{cases}
%     	\ownGhost{\gname_G}{x \mapsto (q, v)} & \text{if $q \in (0, 1]$} \\
%     	\FALSE & \text{otherwise}
%     \end{cases}
% \]
% The view shifts in the specification follow immediately from \ruleref{GhostUpd} and the frame-preserving updates in~\Sref{sec:fheapm}.
% The first implication is immediate from the definition.
% The second implication follows by case distinction on $q_1 + q_2 \in (0, 1]$.


%%% Local Variables:
%%% mode: latex
%%% TeX-master: "iris"
%%% End:
