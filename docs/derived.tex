\section{Derived proof rules and other constructions}

\subsection{Base logic}

We collect here some important and frequently used derived proof rules.
\begin{mathparpagebreakable}
  \infer{}
  {\prop \Ra \propB \proves \prop \wand \propB}

  \infer{}
  {\prop * \Exists\var.\propB \provesIff \Exists\var. \prop * \propB}

  \infer{}
  {\prop * \Exists\var.\propB \proves \Exists\var. \prop * \propB}

  \infer{}
  {\always(\prop*\propB) \provesIff \always\prop * \always\propB}

  \infer{}
  {\always(\prop \Ra \propB) \proves \always\prop \Ra \always\propB}

  \infer{}
  {\always(\prop \wand \propB) \proves \always\prop \wand \always\propB}

  \infer{}
  {\always(\prop \wand \propB) \provesIff \always(\prop \Ra \propB)}

  \infer{}
  {\later(\prop \Ra \propB) \proves \later\prop \Ra \later\propB}

  \infer{}
  {\later(\prop \wand \propB) \proves \later\prop \wand \later\propB}

  \infer
  {\pfctx, \later\prop \proves \prop}
  {\pfctx \proves \prop}
\end{mathparpagebreakable}

\paragraph{Persistent assertions.}
\begin{defn}
  An assertion $\prop$ is \emph{persistent} if $\prop \proves \always\prop$.
\end{defn}

Of course, $\always\prop$ is persistent for any $\prop$.
Furthermore, by the proof rules given above, $t = t'$ as well as $\ownGGhost{\mcore\melt}$ and $\knowInv\iname\prop$ are persistent.
Persistence is preserved by conjunction, disjunction, separating conjunction as well as universal and existential quantification.

In our proofs, we will implicitly add and remove $\always$ from persistent assertions as necessary, and generally treat them like normal, non-linear assumptions.

\paragraph{Timeless assertions.}

We can show that the following additional closure properties hold for timeless assertions:

\begin{mathparpagebreakable}
  \infer
  {\vctx \proves \timeless{\prop} \and \vctx \proves \timeless{\propB}}
  {\vctx \proves \timeless{\prop \land \propB}}

  \infer
  {\vctx \proves \timeless{\prop} \and \vctx \proves \timeless{\propB}}
  {\vctx \proves \timeless{\prop \lor \propB}}

  \infer
  {\vctx \proves \timeless{\prop} \and \vctx \proves \timeless{\propB}}
  {\vctx \proves \timeless{\prop * \propB}}

  \infer
  {\vctx \proves \timeless{\prop}}
  {\vctx \proves \timeless{\always\prop}}
\end{mathparpagebreakable}


\subsection{Program logic}

Hoare triples and view shifts are syntactic sugar for weakest (liberal) preconditions and primitive view shifts, respectively:
\[
\hoare{\prop}{\expr}{\Ret\val.\propB}[\mask] \eqdef \always{(\prop \Ra \wpre{\expr}[\mask]{\lambda\Ret\val.\propB})}
\qquad\qquad
\begin{aligned}
\prop \vs[\mask_1][\mask_2] \propB &\eqdef \always{(\prop \Ra \pvs[\mask_1][\mask_2] {\propB})} \\
\prop \vsE[\mask_1][\mask_2] \propB &\eqdef \prop \vs[\mask_1][\mask_2] \propB \land \propB \vs[\mask2][\mask_1] \prop
\end{aligned}
\]
We write just one mask for a view shift when $\mask_1 = \mask_2$.
Clearly, all of these assertions are persistent.
The convention for omitted masks is similar to the base logic:
An omitted $\mask$ is $\top$ for Hoare triples and $\emptyset$ for view shifts.


\paragraph{View shifts.}
The following rules can be derived for view shifts.

\begin{mathparpagebreakable}
\inferH{vs-update}
  {\melt \mupd \meltsB}
  {\ownGGhost{\melt} \vs \exists \meltB \in \meltsB.\; \ownGGhost{\meltB}}
\and
\inferH{vs-trans}
  {\prop \vs[\mask_1][\mask_2] \propB \and \propB \vs[\mask_2][\mask_3] \propC \and \mask_2 \subseteq \mask_1 \cup \mask_3}
  {\prop \vs[\mask_1][\mask_3] \propC}
\and
\inferH{vs-imp}
  {\always{(\prop \Ra \propB)}}
  {\prop \vs[\emptyset] \propB}
\and
\inferH{vs-mask-frame}
  {\prop \vs[\mask_1][\mask_2] \propB}
  {\prop \vs[\mask_1 \uplus \mask'][\mask_2 \uplus \mask'] \propB}
\and
\inferH{vs-frame}
  {\prop \vs[\mask_1][\mask_2] \propB}
  {\prop * \propC \vs[\mask_1][\mask_2] \propB * \propC}
\and
\inferH{vs-timeless}
  {\timeless{\prop}}
  {\later \prop \vs \prop}
\and
\inferH{vs-allocI}
  {\infinite(\mask)}
  {\later{\prop} \vs[\mask] \exists \iname\in\mask.\; \knowInv{\iname}{\prop}}
\and
\axiomH{vs-openI}
  {\knowInv{\iname}{\prop} \proves \TRUE \vs[\{ \iname \} ][\emptyset] \later \prop}
\and
\axiomH{vs-closeI}
  {\knowInv{\iname}{\prop} \proves \later \prop \vs[\emptyset][\{ \iname \} ] \TRUE }

\inferHB{vs-disj}
  {\prop \vs[\mask_1][\mask_2] \propC \and \propB \vs[\mask_1][\mask_2] \propC}
  {\prop \lor \propB \vs[\mask_1][\mask_2] \propC}
\and
\inferHB{vs-exist}
  {\All \var. (\prop \vs[\mask_1][\mask_2] \propB)}
  {(\Exists \var. \prop) \vs[\mask_1][\mask_2] \propB}
\and
\inferHB{vs-box}
  {\always\propB \proves \prop \vs[\mask_1][\mask_2] \propC}
  {\prop \land \always{\propB} \vs[\mask_1][\mask_2] \propC}
 \and
\inferH{vs-false}
  {}
  {\FALSE \vs[\mask_1][\mask_2] \prop }
\end{mathparpagebreakable}


\paragraph{Hoare triples.}
The following rules can be derived for Hoare triples.

\begin{mathparpagebreakable}
\inferH{Ht-ret}
  {}
  {\hoare{\TRUE}{\valB}{\Ret\val. \val = \valB}[\mask]}
\and
\inferH{Ht-bind}
  {\text{$\lctx$ is a context} \and \hoare{\prop}{\expr}{\Ret\val. \propB}[\mask] \\
   \All \val. \hoare{\propB}{\lctx(\val)}{\Ret\valB.\propC}[\mask]}
  {\hoare{\prop}{\lctx(\expr)}{\Ret\valB.\propC}[\mask]}
\and
\inferH{Ht-csq}
  {\prop \vs \prop' \\
    \hoare{\prop'}{\expr}{\Ret\val.\propB'}[\mask] \\   
   \All \val. \propB' \vs \propB}
  {\hoare{\prop}{\expr}{\Ret\val.\propB}[\mask]}
\and
\inferH{Ht-mask-weaken}
  {\hoare{\prop}{\expr}{\Ret\val. \propB}[\mask]}
  {\hoare{\prop}{\expr}{\Ret\val. \propB}[\mask \uplus \mask']}
\\\\
\inferH{Ht-frame}
  {\hoare{\prop}{\expr}{\Ret\val. \propB}[\mask]}
  {\hoare{\prop * \propC}{\expr}{\Ret\val. \propB * \propC}[\mask]}
\and
\inferH{Ht-frame-step}
  {\hoare{\prop}{\expr}{\Ret\val. \propB}[\mask] \and \toval(\expr) = \bot}
  {\hoare{\prop * \later\propC}{\expr}{\Ret\val. \propB * \propC}[\mask]}
\and
\inferH{Ht-atomic}
  {\prop \vs[\mask \uplus \mask'][\mask] \prop' \\
    \hoare{\prop'}{\expr}{\Ret\val.\propB'}[\mask] \\   
   \All\val. \propB' \vs[\mask][\mask \uplus \mask'] \propB \\
   \physatomic{\expr}
  }
  {\hoare{\prop}{\expr}{\Ret\val.\propB}[\mask \uplus \mask']}
\and
\inferHB{Ht-disj}
  {\hoare{\prop}{\expr}{\Ret\val.\propC}[\mask] \and \hoare{\propB}{\expr}{\Ret\val.\propC}[\mask]}
  {\hoare{\prop \lor \propB}{\expr}{\Ret\val.\propC}[\mask]}
\and
\inferHB{Ht-exist}
  {\All \var. \hoare{\prop}{\expr}{\Ret\val.\propB}[\mask]}
  {\hoare{\Exists \var. \prop}{\expr}{\Ret\val.\propB}[\mask]}
\and
\inferHB{Ht-box}
  {\always\propB \proves \hoare{\prop}{\expr}{\Ret\val.\propC}[\mask]}
  {\hoare{\prop \land \always{\propB}}{\expr}{\Ret\val.\propC}[\mask]}
\and
\inferH{Ht-false}
  {}
  {\hoare{\FALSE}{\expr}{\Ret \val. \prop}[\mask]}
\end{mathparpagebreakable}

\paragraph{Lifting of operational semantics.}
We can derive some specialized forms of the lifting axioms for the operational semantics, as well as some forms that involve view shifts and Hoare triples.

\ralf{Add these.}

\subsection{Global functor and ghost ownership}
\ralf{Describe this.}

% \subsection{Global monoid}

% Hereinafter we assume the global monoid (served up as a parameter to Iris) is obtained from a family of monoids $(M_i)_{i \in I}$ by first applying the construction for finite partial functions to each~(\Sref{sec:fpfunm}), and then applying the product construction~(\Sref{sec:prodm}):
% \[ M \eqdef \prod_{i \in I} \textdom{GhName} \fpfn M_i \]
% We don't care so much about what concretely $\textdom{GhName}$ is, as long as it is countable and infinite.
% We write $\ownGhost{\gname}{\melt : M_i}$ (or just $\ownGhost{\gname}{\melt}$ if $M_i$ is clear from the context) for $\ownGGhost{[i \mapsto [\gname \mapsto \melt]]}$ when $\melt \in \mcarp {M_i}$, and for $\FALSE$ when $\melt = \mzero_{M_i}$.
% In other words, $\ownGhost{\gname}{\melt : M_i}$ asserts that in the current state of monoid $M_i$, the name $\gname$ is allocated and has at least value $\melt$.

% From~\ruleref{FpUpd} and the multiplications and frame-preserving updates in~\Sref{sec:prodm} and~\Sref{sec:fpfunm}, we have the following derived rules.
% \begin{mathpar}
% 	\axiomH{NewGhost}{
% 		\TRUE \vs \Exists\gname. \ownGhost\gname{\melt : M_i}
% 	}
% 	\and
% 	\inferH{GhostUpd}
%     {\melt \mupd_{M_i} B}
%     {\ownGhost\gname{\melt : M_i} \vs \Exists \meltB\in B. \ownGhost\gname{\meltB : M_i}}
%   \and
%   \axiomH{GhostEq}
%     {\ownGhost\gname{\melt : M_i} * \ownGhost\gname{\meltB : M_i} \Lra \ownGhost\gname{\melt\mtimes\meltB : M_i}}

%   \axiomH{GhostUnit}
%     {\TRUE \Ra \ownGhost{\gname}{\munit : M_i}}

%   \axiomH{GhostZero}
%     {\ownGhost\gname{\mzero : M_i} \Ra \FALSE}

%   \axiomH{GhostTimeless}
%     {\timeless{\ownGhost\gname{\melt : M_i}}}
% \end{mathpar}

\subsection{Invariant identifier namespaces}

Let $\namesp \ni \textlog{InvNamesp} \eqdef \textlog{list}(\textlog{InvName})$ be the type of \emph{namespaces} for invariant names.
Notice that there is an injection $\textlog{namesp\_inj}: \textlog{InvNamesp} \ra \textlog{InvName}$.
Whenever needed (in particular, for masks at view shifts and Hoare triples), we coerce $\namesp$ to its suffix-closure: \[\namecl\namesp \eqdef \setComp{\iname}{\Exists \namesp'. \iname = \textlog{namesp\_inj}(\namesp' \dplus \namesp)}\]
We use the notation $\namesp.\iname$ for the namespace $[\iname] \dplus \namesp$.

We define the inclusion relation on namespaces as $\namesp_1 \sqsubseteq \namesp_2 \Lra \Exists \namesp_3. \namesp_2 = \namesp_3 \dplus \namesp_1$, \ie $\namesp_1$ is a suffix of $\namesp_2$.
We have that $\namesp_1 \sqsubseteq \namesp_2 \Ra \namecl\namesp_2 \subseteq \namecl\namesp_1$.

Similarly, we define $\namesp_1 \sep \namesp_2 \eqdef   \Exists \namesp_1', \namesp_2'. \namesp_1' \sqsubseteq \namesp_1 \land \namesp_2' \sqsubseteq \namesp_2 \land |\namesp_1'| = |\namesp_2'| \land \namesp_1' \neq \namesp_2'$, \ie there exists a distinguishing suffix.
We have that $\namesp_1 \sep \namesp_2 \Ra \namecl\namesp_2 \sep \namecl\namesp_1$, and furthermore $\iname_1 \neq \iname_2 \Ra \namesp.\iname_1 \sep \namesp.\iname_2$.

We will overload the usual Iris notation for invariant assertions in the following:
\[ \knowInv\namesp\prop \eqdef \Exists \iname \in \namecl\namesp. \knowInv\iname{\prop} \]
We can now derive the following rules for this derived form of the invariant assertion:
\begin{mathpar}
  \axiom{\knowInv\namesp\prop \proves \always\knowInv\namesp\prop}

  \axiom{\later\prop \proves \pvs[\namesp] \knowInv\namesp\prop}

  \infer{\physatomic{\expr} \and \namesp \subseteq \mask \and
    \pfctx \proves \knowInv\namesp\prop \and
    \pfctx \proves \later\prop \wand \wpre\expr[\mask \setminus \namesp]{\Ret\val.\later\prop * \propB}}
  {\pfctx \proves \wpre\expr[\mask]{\Ret\val.\propB}}

  \infer{\namesp \subseteq \mask \and
    \pfctx \proves \knowInv\namesp\prop \and
    \pfctx \proves \later\prop \wand \pvs[\mask \setminus \namesp]{\later\prop * \propB}}
  {\pfctx \proves \pvs[\mask]{\propB}}

  \infer{\physatomic{\expr} \and \namesp \subseteq \mask \and
    \hoare{\later\prop*\propB}\expr{\Ret\val.\later\prop*\propC}[\mask \setminus \namesp]}
  {\knowInv\namesp\prop \proves \hoare\propB\expr{\Ret\val.\propC}[\mask]}

  \infer{\namesp \subseteq \mask \and
    \later\prop*\propB \vs[\mask \setminus \namesp] \later\prop*\propC}
  {\knowInv\namesp\prop \proves \propB \vs[\mask] \propC}
\end{mathpar}

% \subsection{STSs with interpretation}\label{sec:stsinterp}

% Building on \Sref{sec:stsmon}, after constructing the monoid $\STSMon{\STSS}$ for a particular STS, we can use an invariant to tie an interpretation, $\pred : \STSS \to \Prop$, to the STS's current state, recovering CaReSL-style reasoning~\cite{caresl}.

% An STS invariant asserts authoritative ownership of an STS's current state and that state's interpretation:
% \begin{align*}
%   \STSInv(\STSS, \pred, \gname) \eqdef{}& \Exists s \in \STSS. \ownGhost{\gname}{(s, \STSS, \emptyset):\STSMon{\STSS}} * \pred(s) \\
%   \STS(\STSS, \pred, \gname, \iname) \eqdef{}& \knowInv{\iname}{\STSInv(\STSS, \pred, \gname)}
% \end{align*}

% We can specialize \ruleref{NewInv}, \ruleref{InvOpen}, and \ruleref{InvClose} to STS invariants:
% \begin{mathpar}
%  \inferH{NewSts}
%   {\infinite(\mask)}
%   {\later\pred(s) \vs[\mask] \Exists \iname \in \mask, \gname.   \STS(\STSS, \pred, \gname, \iname) * \ownGhost{\gname}{(s, \STST \setminus \STSL(s)) : \STSMon{\STSS}}}
%  \and
%  \axiomH{StsOpen}
%   {  \STS(\STSS, \pred, \gname, \iname) \vdash \ownGhost{\gname}{(s_0, T) : \STSMon{\STSS}} \vsE[\{\iname\}][\emptyset] \Exists s\in \upclose(\{s_0\}, T). \later\pred(s) * \ownGhost{\gname}{(s, \upclose(\{s_0\}, T), T):\STSMon{\STSS}}}
%  \and
%  \axiomH{StsClose}
%   {  \STS(\STSS, \pred, \gname, \iname), (s, T) \ststrans (s', T')  \proves \later\pred(s') * \ownGhost{\gname}{(s, S, T):\STSMon{\STSS}} \vs[\emptyset][\{\iname\}] \ownGhost{\gname}{(s', T') : \STSMon{\STSS}} }
% \end{mathpar}
% \begin{proof}
% \ruleref{NewSts} uses \ruleref{NewGhost} to allocate $\ownGhost{\gname}{(s, \upclose(s, T), T) : \STSMon{\STSS}}$ where $T \eqdef \STST \setminus \STSL(s)$, and \ruleref{NewInv}.

% \ruleref{StsOpen} just uses \ruleref{InvOpen} and \ruleref{InvClose} on $\iname$, and the monoid equality $(s, \upclose(\{s_0\}, T), T) = (s, \STSS, \emptyset) \mtimes (\munit, \upclose(\{s_0\}, T), T)$.

% \ruleref{StsClose} applies \ruleref{StsStep} and \ruleref{InvClose}.
% \end{proof}

% Using these view shifts, we can prove STS variants of the invariant rules \ruleref{Inv} and \ruleref{VSInv}~(compare the former to CaReSL's island update rule~\cite{caresl}):
% \begin{mathpar}
%  \inferH{Sts}
%   {\All s \in \upclose(\{s_0\}, T). \hoare{\later\pred(s) * P}{\expr}{\Ret \val. \Exists s', T'. (s, T) \ststrans (s', T') * \later\pred(s') * Q}[\mask]
%    \and \physatomic{\expr}}
%   {  \STS(\STSS, \pred, \gname, \iname) \vdash \hoare{\ownGhost{\gname}{(s_0, T):\STSMon{\STSS}} * P}{\expr}{\Ret \val. \Exists s', T'. \ownGhost{\gname}{(s', T'):\STSMon{\STSS}} * Q}[\mask \uplus \{\iname\}]}
%  \and
%  \inferH{VSSts}
%   {\forall s \in \upclose(\{s_0\}, T).\; \later\pred(s) * P \vs[\mask_1][\mask_2] \exists s', T'.\; (s, T) \ststrans (s', T') * \later\pred(s') * Q}
%   {  \STS(\STSS, \pred, \gname, \iname) \vdash \ownGhost{\gname}{(s_0, T):\STSMon{\STSS}} * P \vs[\mask_1 \uplus \{\iname\}][\mask_2 \uplus \{\iname\}] \Exists s', T'. \ownGhost{\gname}{(s', T'):\STSMon{\STSS}} * Q}
% \end{mathpar}

% \begin{proof}[Proof of \ruleref{Sts}]\label{pf:sts}
%  We have to show
%  \[\hoare{\ownGhost{\gname}{(s_0, T):\STSMon{\STSS}} * P}{\expr}{\Ret \val. \Exists s', T'. \ownGhost{\gname}{(s', T'):\STSMon{\STSS}} * Q}[\mask \uplus \{\iname\}]\]
%  where $\val$, $s'$, $T'$ are free in $Q$.
 
%  First, by \ruleref{ACsq} with \ruleref{StsOpen} and \ruleref{StsClose} (after moving $(s, T) \ststrans (s', T')$ into the view shift using \ruleref{VSBoxOut}), it suffices to show
%  \[\hoareV{\Exists s\in \upclose(\{s_0\}, T). \later\pred(s) * \ownGhost{\gname}{(s, \upclose(\{s_0\}, T), T)} * P}{\expr}{\Ret \val. \Exists s, T, S, s', T'. (s, T) \ststrans (s', T') * \later\pred(s') * \ownGhost{\gname}{(s, S, T):\STSMon{\STSS}} * Q(\val, s', T')}[\mask]\]

%  Now, use \ruleref{Exist} to move the $s$ from the precondition into the context and use \ruleref{Csq} to (i)~fix the $s$ and $T$ in the postcondition to be the same as in the precondition, and (ii)~fix $S \eqdef \upclose(\{s_0\}, T)$.
%  It remains to show:
%  \[\hoareV{s\in \upclose(\{s_0\}, T) * \later\pred(s) * \ownGhost{\gname}{(s, \upclose(\{s_0\}, T), T)} * P}{\expr}{\Ret \val. \Exists s', T'. (s, T) \ststrans (s', T') * \later\pred(s') * \ownGhost{\gname}{(s, \upclose(\{s_0\}, T), T)} * Q(\val, s', T')}[\mask]\]
 
%  Finally, use \ruleref{BoxOut} to move $s\in \upclose(\{s_0\}, T)$ into the context, and \ruleref{Frame} on $\ownGhost{\gname}{(s, \upclose(\{s_0\}, T), T)}$:
%  \[s\in \upclose(\{s_0\}, T) \vdash \hoare{\later\pred(s) * P}{\expr}{\Ret \val. \Exists s', T'. (s, T) \ststrans (s', T') * \later\pred(s') * Q(\val, s', T')}[\mask]\]
 
%  This holds by our premise.
% \end{proof}

% % \begin{proof}[Proof of \ruleref{VSSts}]
% % This is similar to above, so we only give the proof in short notation:

% % \hproof{%
% % 	Context: $\knowInv\iname{\STSInv(\STSS, \pred, \gname)}$ \\
% % 	\pline[\mask_1 \uplus \{\iname\}]{
% % 		\ownGhost\gname{(s_0, T)} * P
% % 	} \\
% % 	\pline[\mask_1]{%
% % 		\Exists s. \later\pred(s) * \ownGhost\gname{(s, S, T)} * P
% % 	} \qquad by \ruleref{StsOpen} \\
% % 	Context: $s \in S \eqdef \upclose(\{s_0\}, T)$ \\
% % 	\pline[\mask_2]{%
% % 		 \Exists s', T'. \later\pred(s') * Q(s', T') * \ownGhost\gname{(s, S, T)}
% % 	} \qquad by premiss \\
% % 	Context: $(s, T) \ststrans (s', T')$ \\
% % 	\pline[\mask_2 \uplus \{\iname\}]{
% % 		\ownGhost\gname{(s', T')} * Q(s', T')
% % 	} \qquad by \ruleref{StsClose}
% % }
% % \end{proof}

% \subsection{Authoritative monoids with interpretation}\label{sec:authinterp}

% Building on \Sref{sec:auth}, after constructing the monoid $\auth{M}$ for a cancellative monoid $M$, we can tie an interpretation, $\pred : \mcarp{M} \to \Prop$, to the authoritative element of $M$, recovering reasoning that is close to the sharing rule in~\cite{krishnaswami+:icfp12}.

% Let $\pred_\bot$ be the extension of $\pred$ to $\mcar{M}$ with $\pred_\bot(\mzero) = \FALSE$.
% Now define
% \begin{align*}
%   \AuthInv(M, \pred, \gname) \eqdef{}& \exists \melt \in \mcar{M}.\; \ownGhost{\gname}{\authfull \melt:\auth{M}} * \pred_\bot(\melt) \\
%   \Auth(M, \pred, \gname, \iname) \eqdef{}& M~\textlog{cancellative} \land \knowInv{\iname}{\AuthInv(M, \pred, \gname)}
% \end{align*}

% The frame-preserving updates for $\auth{M}$ gives rise to the following view shifts:
% \begin{mathpar}
%  \inferH{NewAuth}
%   {\infinite(\mask) \and M~\textlog{cancellative}}
%   {\later\pred_\bot(a) \vs[\mask] \exists \iname \in \mask, \gname.\; \Auth(M, \pred, \gname, \iname) * \ownGhost{\gname}{\authfrag a : \auth{M}}}
%  \and
%  \axiomH{AuthOpen}
%   {\Auth(M, \pred, \gname, \iname) \vdash \ownGhost{\gname}{\authfrag \melt : \auth{M}} \vsE[\{\iname\}][\emptyset] \exists \melt_f.\; \later\pred_\bot(\melt \mtimes \melt_f) * \ownGhost{\gname}{\authfull \melt \mtimes \melt_f, \authfrag a:\auth{M}}}
%  \and
%  \axiomH{AuthClose}
%   {\Auth(M, \pred, \gname, \iname) \vdash \later\pred_\bot(\meltB \mtimes \melt_f) * \ownGhost{\gname}{\authfull a \mtimes \melt_f, \authfrag a:\auth{M}} \vs[\emptyset][\{\iname\}] \ownGhost{\gname}{\authfrag \meltB : \auth{M}} }
% \end{mathpar}

% These view shifts in turn can be used to prove variants of the invariant rules:
% \begin{mathpar}
%  \inferH{Auth}
%   {\forall \melt_f.\; \hoare{\later\pred_\bot(a \mtimes \melt_f) * P}{\expr}{\Ret\val. \exists \meltB.\; \later\pred_\bot(\meltB\mtimes \melt_f) * Q}[\mask]
%    \and \physatomic{\expr}}
%   {\Auth(M, \pred, \gname, \iname) \vdash \hoare{\ownGhost{\gname}{\authfrag a:\auth{M}} * P}{\expr}{\Ret\val. \exists \meltB.\; \ownGhost{\gname}{\authfrag \meltB:\auth{M}} * Q}[\mask \uplus \{\iname\}]}
%  \and
%  \inferH{VSAuth}
%   {\forall \melt_f.\; \later\pred_\bot(a \mtimes \melt_f) * P \vs[\mask_1][\mask_2] \exists \meltB.\; \later\pred_\bot(\meltB \mtimes \melt_f) * Q(\meltB)}
%   {\Auth(M, \pred, \gname, \iname) \vdash
%    \ownGhost{\gname}{\authfrag a:\auth{M}} * P \vs[\mask_1 \uplus \{\iname\}][\mask_2 \uplus \{\iname\}]
%    \exists \meltB.\; \ownGhost{\gname}{\authfrag \meltB:\auth{M}} * Q(\meltB)}
% \end{mathpar}


% \subsection{Ghost heap}
% \label{sec:ghostheap}%
% FIXME use the finmap provided by the global ghost ownership, instead of adding our own
% We define a simple ghost heap with fractional permissions.
% Some modules require a few ghost names per module instance to properly manage ghost state, but would like to expose to clients a single logical name (avoiding clutter).
% In such cases we use these ghost heaps.

% We seek to implement the following interface:
% \newcommand{\GRefspecmaps}{\textsf{GMapsTo}}%
% \begin{align*}
%  \exists& {\fgmapsto[]} : \textsort{Val} \times \mathbb{Q}_{>} \times \textsort{Val} \ra \textsort{Prop}.\;\\
%   & \All x, q, v. x \fgmapsto[q] v \Ra x \fgmapsto[q] v \land q \in (0, 1] \\
%   &\forall x, q_1, q_2, v, w.\; x \fgmapsto[q_1] v * x \fgmapsto[q_2] w \Leftrightarrow x \fgmapsto[q_1 + q_2] v * v = w\\
%   & \forall v.\; \TRUE \vs[\emptyset] \exists x.\; x \fgmapsto[1] v \\
%   & \forall x, v, w.\; x \fgmapsto[1] v \vs[\emptyset] x \fgmapsto[1] w
% \end{align*}
% We write $x \fgmapsto v$ for $\exists q.\; x \fgmapsto[q] v$ and $x \gmapsto v$ for $x \fgmapsto[1] v$.
% Note that $x \fgmapsto v$ is duplicable but cannot be boxed (as it depends on resources); \ie we have $x \fgmapsto v \Lra x \fgmapsto v * x \fgmapsto v$ but not $x \fgmapsto v \Ra \always x \fgmapsto v$.

% To implement this interface, allocate an instance $\gname_G$ of $\FHeap(\textdom{Val})$ and define
% \[
% 	x \fgmapsto[q] v \eqdef
% 	  \begin{cases}
%     	\ownGhost{\gname_G}{x \mapsto (q, v)} & \text{if $q \in (0, 1]$} \\
%     	\FALSE & \text{otherwise}
%     \end{cases}
% \]
% The view shifts in the specification follow immediately from \ruleref{GhostUpd} and the frame-preserving updates in~\Sref{sec:fheapm}.
% The first implication is immediate from the definition.
% The second implication follows by case distinction on $q_1 + q_2 \in (0, 1]$.


%%% Local Variables:
%%% mode: latex
%%% TeX-master: "iris"
%%% End:
