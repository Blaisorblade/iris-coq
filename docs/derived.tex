\section{Derived constructions}

In this section we describe some constructions that we will use throughout the rest of the appendix.

\subsection{Global monoid}

Hereinafter we assume the global monoid (served up as a parameter to Iris) is obtained from a family of monoids $(M_i)_{i \in I}$ by first applying the construction for finite partial functions to each~(\Sref{sec:fpfunm}), and then applying the product construction~(\Sref{sec:prodm}):
\[ M \eqdef \prod_{i \in I} \fpfunm{\textdom{GhName}}{M_i} \]
We don't care so much about what concretely $\textdom{GhName}$ is, as long as it is countable and infinite.
We write $\ownGhost{\gname}{\melt : M_i}$ (or just $\ownGhost{\gname}{\melt}$ if $M_i$ is clear from the context) for $\ownGGhost{[i \mapsto [\gname \mapsto \melt]]}$ when $\melt \in \mcarp {M_i}$, and for $\FALSE$ when $\melt = \mzero_{M_i}$.
In other words, $\ownGhost{\gname}{\melt : M_i}$ asserts that in the current state of monoid $M_i$, the name $\gname$ is allocated and has at least value $\melt$.

From~\ruleref{FpUpd} and the multiplications and frame-preserving updates in~\Sref{sec:prodm} and~\Sref{sec:fpfunm}, we have the following derived rules.
\begin{mathpar}
	\axiomH{NewGhost}{
		\TRUE \vs \Exists\gname. \ownGhost\gname{\melt : M_i}
	}
	\and
	\inferH{GhostUpd}
    {\melt \mupd_{M_i} B}
    {\ownGhost\gname{\melt : M_i} \vs \Exists \meltB\in B. \ownGhost\gname{\meltB : M_i}}
  \and
  \axiomH{GhostEq}
    {\ownGhost\gname{\melt : M_i} * \ownGhost\gname{\meltB : M_i} \Lra \ownGhost\gname{\melt\mtimes\meltB : M_i}}

  \axiomH{GhostUnit}
    {\TRUE \Ra \ownGhost{\gname}{\munit : M_i}}

  \axiomH{GhostZero}
    {\ownGhost\gname{\mzero : M_i} \Ra \FALSE}

  \axiomH{GhostTimeless}
    {\timeless{\ownGhost\gname{\melt : M_i}}}
\end{mathpar}

\subsection{STSs with interpretation}\label{sec:stsinterp}

Building on \Sref{sec:stsmon}, after constructing the monoid $\STSMon{\STSS}$ for a particular STS, we can use an invariant to tie an interpretation, $\pred : \STSS \to \Prop$, to the STS's current state, recovering CaReSL-style reasoning~\cite{caresl}.

An STS invariant asserts authoritative ownership of an STS's current state and that state's interpretation:
\begin{align*}
  \STSInv(\STSS, \pred, \gname) \eqdef{}& \Exists s \in \STSS. \ownGhost{\gname}{(s, \STSS, \emptyset):\STSMon{\STSS}} * \pred(s) \\
  \STS(\STSS, \pred, \gname, \iname) \eqdef{}& \knowInv{\iname}{\STSInv(\STSS, \pred, \gname)}
\end{align*}

We can specialize \ruleref{NewInv}, \ruleref{InvOpen}, and \ruleref{InvClose} to STS invariants:
\begin{mathpar}
 \inferH{NewSts}
  {\infinite(\mask)}
  {\later\pred(s) \vs[\mask] \Exists \iname \in \mask, \gname.   \STS(\STSS, \pred, \gname, \iname) * \ownGhost{\gname}{(s, \STST \setminus \STSL(s)) : \STSMon{\STSS}}}
 \and
 \axiomH{StsOpen}
  {  \STS(\STSS, \pred, \gname, \iname) \vdash \ownGhost{\gname}{(s_0, T) : \STSMon{\STSS}} \vsE[\{\iname\}][\emptyset] \Exists s\in \upclose(\{s_0\}, T). \later\pred(s) * \ownGhost{\gname}{(s, \upclose(\{s_0\}, T), T):\STSMon{\STSS}}}
 \and
 \axiomH{StsClose}
  {  \STS(\STSS, \pred, \gname, \iname), (s, T) \ststrans (s', T')  \proves \later\pred(s') * \ownGhost{\gname}{(s, S, T):\STSMon{\STSS}} \vs[\emptyset][\{\iname\}] \ownGhost{\gname}{(s', T') : \STSMon{\STSS}} }
\end{mathpar}
\begin{proof}
\ruleref{NewSts} uses \ruleref{NewGhost} to allocate $\ownGhost{\gname}{(s, \upclose(s, T), T) : \STSMon{\STSS}}$ where $T \eqdef \STST \setminus \STSL(s)$, and \ruleref{NewInv}.

\ruleref{StsOpen} just uses \ruleref{InvOpen} and \ruleref{InvClose} on $\iname$, and the monoid equality $(s, \upclose(\{s_0\}, T), T) = (s, \STSS, \emptyset) \mtimes (\munit, \upclose(\{s_0\}, T), T)$.

\ruleref{StsClose} applies \ruleref{StsStep} and \ruleref{InvClose}.
\end{proof}

Using these view shifts, we can prove STS variants of the invariant rules \ruleref{Inv} and \ruleref{VSInv}~(compare the former to CaReSL's island update rule~\cite{caresl}):
\begin{mathpar}
 \inferH{Sts}
  {\All s \in \upclose(\{s_0\}, T). \hoare{\later\pred(s) * P}{\expr}{\Ret \val. \Exists s', T'. (s, T) \ststrans (s', T') * \later\pred(s') * Q}[\mask]
   \and \physatomic{\expr}}
  {  \STS(\STSS, \pred, \gname, \iname) \vdash \hoare{\ownGhost{\gname}{(s_0, T):\STSMon{\STSS}} * P}{\expr}{\Ret \val. \Exists s', T'. \ownGhost{\gname}{(s', T'):\STSMon{\STSS}} * Q}[\mask \uplus \{\iname\}]}
 \and
 \inferH{VSSts}
  {\forall s \in \upclose(\{s_0\}, T).\; \later\pred(s) * P \vs[\mask_1][\mask_2] \exists s', T'.\; (s, T) \ststrans (s', T') * \later\pred(s') * Q}
  {  \STS(\STSS, \pred, \gname, \iname) \vdash \ownGhost{\gname}{(s_0, T):\STSMon{\STSS}} * P \vs[\mask_1 \uplus \{\iname\}][\mask_2 \uplus \{\iname\}] \Exists s', T'. \ownGhost{\gname}{(s', T'):\STSMon{\STSS}} * Q}
\end{mathpar}

\begin{proof}[Proof of \ruleref{Sts}]\label{pf:sts}
 We have to show
 \[\hoare{\ownGhost{\gname}{(s_0, T):\STSMon{\STSS}} * P}{\expr}{\Ret \val. \Exists s', T'. \ownGhost{\gname}{(s', T'):\STSMon{\STSS}} * Q}[\mask \uplus \{\iname\}]\]
 where $\val$, $s'$, $T'$ are free in $Q$.
 
 First, by \ruleref{ACsq} with \ruleref{StsOpen} and \ruleref{StsClose} (after moving $(s, T) \ststrans (s', T')$ into the view shift using \ruleref{VSBoxOut}), it suffices to show
 \[\hoareV{\Exists s\in \upclose(\{s_0\}, T). \later\pred(s) * \ownGhost{\gname}{(s, \upclose(\{s_0\}, T), T)} * P}{\expr}{\Ret \val. \Exists s, T, S, s', T'. (s, T) \ststrans (s', T') * \later\pred(s') * \ownGhost{\gname}{(s, S, T):\STSMon{\STSS}} * Q(\val, s', T')}[\mask]\]

 Now, use \ruleref{Exist} to move the $s$ from the precondition into the context and use \ruleref{Csq} to (i)~fix the $s$ and $T$ in the postcondition to be the same as in the precondition, and (ii)~fix $S \eqdef \upclose(\{s_0\}, T)$.
 It remains to show:
 \[\hoareV{s\in \upclose(\{s_0\}, T) * \later\pred(s) * \ownGhost{\gname}{(s, \upclose(\{s_0\}, T), T)} * P}{\expr}{\Ret \val. \Exists s', T'. (s, T) \ststrans (s', T') * \later\pred(s') * \ownGhost{\gname}{(s, \upclose(\{s_0\}, T), T)} * Q(\val, s', T')}[\mask]\]
 
 Finally, use \ruleref{BoxOut} to move $s\in \upclose(\{s_0\}, T)$ into the context, and \ruleref{Frame} on $\ownGhost{\gname}{(s, \upclose(\{s_0\}, T), T)}$:
 \[s\in \upclose(\{s_0\}, T) \vdash \hoare{\later\pred(s) * P}{\expr}{\Ret \val. \Exists s', T'. (s, T) \ststrans (s', T') * \later\pred(s') * Q(\val, s', T')}[\mask]\]
 
 This holds by our premise.
\end{proof}

% \begin{proof}[Proof of \ruleref{VSSts}]
% This is similar to above, so we only give the proof in short notation:

% \hproof{%
% 	Context: $\knowInv\iname{\STSInv(\STSS, \pred, \gname)}$ \\
% 	\pline[\mask_1 \uplus \{\iname\}]{
% 		\ownGhost\gname{(s_0, T)} * P
% 	} \\
% 	\pline[\mask_1]{%
% 		\Exists s. \later\pred(s) * \ownGhost\gname{(s, S, T)} * P
% 	} \qquad by \ruleref{StsOpen} \\
% 	Context: $s \in S \eqdef \upclose(\{s_0\}, T)$ \\
% 	\pline[\mask_2]{%
% 		 \Exists s', T'. \later\pred(s') * Q(s', T') * \ownGhost\gname{(s, S, T)}
% 	} \qquad by premiss \\
% 	Context: $(s, T) \ststrans (s', T')$ \\
% 	\pline[\mask_2 \uplus \{\iname\}]{
% 		\ownGhost\gname{(s', T')} * Q(s', T')
% 	} \qquad by \ruleref{StsClose}
% }
% \end{proof}

\subsection{Authoritative monoids with interpretation}\label{sec:authinterp}

Building on \Sref{sec:auth}, after constructing the monoid $\auth{M}$ for a cancellative monoid $M$, we can tie an interpretation, $\pred : \mcarp{M} \to \Prop$, to the authoritative element of $M$, recovering reasoning that is close to the sharing rule in~\cite{krishnaswami+:icfp12}.

Let $\pred_\bot$ be the extension of $\pred$ to $\mcar{M}$ with $\pred_\bot(\mzero) = \FALSE$.
Now define
\begin{align*}
  \AuthInv(M, \pred, \gname) \eqdef{}& \exists \melt \in \mcar{M}.\; \ownGhost{\gname}{\authfull \melt:\auth{M}} * \pred_\bot(\melt) \\
  \Auth(M, \pred, \gname, \iname) \eqdef{}& M~\textlog{cancellative} \land \knowInv{\iname}{\AuthInv(M, \pred, \gname)}
\end{align*}

The frame-preserving updates for $\auth{M}$ gives rise to the following view shifts:
\begin{mathpar}
 \inferH{NewAuth}
  {\infinite(\mask) \and M~\textlog{cancellative}}
  {\later\pred_\bot(a) \vs[\mask] \exists \iname \in \mask, \gname.\; \Auth(M, \pred, \gname, \iname) * \ownGhost{\gname}{\authfrag a : \auth{M}}}
 \and
 \axiomH{AuthOpen}
  {\Auth(M, \pred, \gname, \iname) \vdash \ownGhost{\gname}{\authfrag \melt : \auth{M}} \vsE[\{\iname\}][\emptyset] \exists \melt_f.\; \later\pred_\bot(\melt \mtimes \melt_f) * \ownGhost{\gname}{\authfull \melt \mtimes \melt_f, \authfrag a:\auth{M}}}
 \and
 \axiomH{AuthClose}
  {\Auth(M, \pred, \gname, \iname) \vdash \later\pred_\bot(\meltB \mtimes \melt_f) * \ownGhost{\gname}{\authfull a \mtimes \melt_f, \authfrag a:\auth{M}} \vs[\emptyset][\{\iname\}] \ownGhost{\gname}{\authfrag \meltB : \auth{M}} }
\end{mathpar}

These view shifts in turn can be used to prove variants of the invariant rules:
\begin{mathpar}
 \inferH{Auth}
  {\forall \melt_f.\; \hoare{\later\pred_\bot(a \mtimes \melt_f) * P}{\expr}{\Ret\val. \exists \meltB.\; \later\pred_\bot(\meltB\mtimes \melt_f) * Q}[\mask]
   \and \physatomic{\expr}}
  {\Auth(M, \pred, \gname, \iname) \vdash \hoare{\ownGhost{\gname}{\authfrag a:\auth{M}} * P}{\expr}{\Ret\val. \exists \meltB.\; \ownGhost{\gname}{\authfrag \meltB:\auth{M}} * Q}[\mask \uplus \{\iname\}]}
 \and
 \inferH{VSAuth}
  {\forall \melt_f.\; \later\pred_\bot(a \mtimes \melt_f) * P \vs[\mask_1][\mask_2] \exists \meltB.\; \later\pred_\bot(\meltB \mtimes \melt_f) * Q(\meltB)}
  {\Auth(M, \pred, \gname, \iname) \vdash
   \ownGhost{\gname}{\authfrag a:\auth{M}} * P \vs[\mask_1 \uplus \{\iname\}][\mask_2 \uplus \{\iname\}]
   \exists \meltB.\; \ownGhost{\gname}{\authfrag \meltB:\auth{M}} * Q(\meltB)}
\end{mathpar}


\subsection{Ghost heap}
\label{sec:ghostheap}%

We define a simple ghost heap with fractional permissions.
Some modules require a few ghost names per module instance to properly manage ghost state, but would like to expose to clients a single logical name (avoiding clutter).
In such cases we use these ghost heaps.

We seek to implement the following interface:
\newcommand{\GRefspecmaps}{\textsf{GMapsTo}}%
\begin{align*}
 \exists& {\fgmapsto[]} : \textsort{Val} \times \mathbb{Q}_{>} \times \textsort{Val} \ra \textsort{Prop}.\;\\
  & \All x, q, v. x \fgmapsto[q] v \Ra x \fgmapsto[q] v \land q \in (0, 1] \\
  &\forall x, q_1, q_2, v, w.\; x \fgmapsto[q_1] v * x \fgmapsto[q_2] w \Leftrightarrow x \fgmapsto[q_1 + q_2] v * v = w\\
  & \forall v.\; \TRUE \vs[\emptyset] \exists x.\; x \fgmapsto[1] v \\
  & \forall x, v, w.\; x \fgmapsto[1] v \vs[\emptyset] x \fgmapsto[1] w
\end{align*}
We write $x \fgmapsto v$ for $\exists q.\; x \fgmapsto[q] v$ and $x \gmapsto v$ for $x \fgmapsto[1] v$.
Note that $x \fgmapsto v$ is duplicable but cannot be boxed (as it depends on resources); \ie we have $x \fgmapsto v \Lra x \fgmapsto v * x \fgmapsto v$ but not $x \fgmapsto v \Ra \always x \fgmapsto v$.

To implement this interface, allocate an instance $\gname_G$ of $\FHeap(\textdom{Val})$ and define
\[
	x \fgmapsto[q] v \eqdef
	  \begin{cases}
    	\ownGhost{\gname_G}{x \mapsto (q, v)} & \text{if $q \in (0, 1]$} \\
    	\FALSE & \text{otherwise}
    \end{cases}
\]
The view shifts in the specification follow immediately from \ruleref{GhostUpd} and the frame-preserving updates in~\Sref{sec:fheapm}.
The first implication is immediate from the definition.
The second implication follows by case distinction on $q_1 + q_2 \in (0, 1]$.


%%% Local Variables:
%%% mode: latex
%%% TeX-master: "iris"
%%% End:
