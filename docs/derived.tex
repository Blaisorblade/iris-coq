\section{Derived Program logic}\label{sec:proglog}

Hoare triples and view shifts are syntactic sugar for weakest (liberal) preconditions and primitive view shifts, respectively:
\[
\hoare{\prop}{\expr}{\Ret\val.\propB}[\mask] \eqdef \always{(\prop \Ra \dynA{\expr}{\lambda\Ret\val.\propB}{\mask})}
\qquad\qquad
\begin{aligned}
\prop \vs[\mask_1][\mask_2] \propB &\eqdef \always{(\prop \Ra \pvsA{\propB}{\mask_1}{\mask_2})} \\
\prop \vsE[\mask_1][\mask_2] \propB &\eqdef \prop \vs[\mask_1][\mask_2] \propB \land \propB \vs[\mask2][\mask_1] \prop
\end{aligned}
\]
We write just one mask for a view shift when $\mask_1 = \mask_2$.
The convention for omitted masks is generous:
An omitted $\mask$ is $\top$ for Hoare triples and $\emptyset$ for view shifts.

% PDS: We're repeating ourselves. We gave Γ conventions and we're about to give Θ conventions. Also, the scope of "Below" is unclear.
% Below, we implicitly assume the same context for all judgements which don't have an explicit context at \emph{all} pre-conditions \emph{and} the conclusion.

Henceforward, we implicitly assume a proof context, $\pfctx$, is added to every constituent of the rules.
Generally, this is an arbitrary proof context.
We write $\provesalways$ to denote judgments that can only be extended with a boxed proof context.

\ralf{Give the actual base rules from the Coq development instead}

\subsection{Hoare triples}
\begin{mathpar}
\inferH{Ret}
  {}
  {\hoare{\TRUE}{\valB}{\Ret\val. \val = \valB}[\mask]}
\and
\inferH{Bind}
  {\hoare{\prop}{\expr}{\Ret\val. \propB}[\mask] \\
   \All \val. \hoare{\propB}{K[\val]}{\Ret\valB.\propC}[\mask]}
  {\hoare{\prop}{K[\expr]}{\Ret\valB.\propC}[\mask]}
\and
\inferH{Csq}
  {\prop \vs \prop' \\
    \hoare{\prop'}{\expr}{\Ret\val.\propB'}[\mask] \\   
   \All \val. \propB' \vs \propB}
  {\hoare{\prop}{\expr}{\Ret\val.\propB}[\mask]}
\and
\inferH{Frame}
  {\hoare{\prop}{\expr}{\Ret\val. \propB}[\mask]}
  {\hoare{\prop * \propC}{\expr}{\Ret\val. \propB * \propC}[\mask \uplus \mask']}
\and
\inferH{AFrame}
  {\hoare{\prop}{\expr}{\Ret\val. \propB}[\mask] \and \text{$\expr$ not a value}
  }
  {\hoare{\prop * \later\propC}{\expr}{\Ret\val. \propB * \propC}[\mask \uplus \mask']}
% \and
% \inferH{Fork}
%   {\hoare{\prop}{\expr}{\Ret\any. \TRUE}[\top]}
%   {\hoare{\later\prop * \later\propB}{\fork{\expr}}{\Ret\val. \val = \textsf{fRet} \land \propB}[\mask]}
\and
\inferH{ACsq}
  {\prop \vs[\mask \uplus \mask'][\mask] \prop' \\
    \hoare{\prop'}{\expr}{\Ret\val.\propB'}[\mask] \\   
   \All\val. \propB' \vs[\mask][\mask \uplus \mask'] \propB \\
   \physatomic{\expr}
  }
  {\hoare{\prop}{\expr}{\Ret\val.\propB}[\mask \uplus \mask']}
\end{mathpar}

\subsection{View shifts}

\begin{mathpar}
\inferH{NewInv}
  {\infinite(\mask)}
  {\later{\prop} \vs[\mask] \exists \iname\in\mask.\; \knowInv{\iname}{\prop}}
\and
\inferH{FpUpd}
  {\melt \mupd \meltsB}
  {\ownGGhost{\melt} \vs \exists \meltB \in \meltsB.\; \ownGGhost{\meltB}}
\and
\inferH{VSTrans}
  {\prop \vs[\mask_1][\mask_2] \propB \and \propB \vs[\mask_2][\mask_3] \propC \and \mask_2 \subseteq \mask_1 \cup \mask_3}
  {\prop \vs[\mask_1][\mask_3] \propC}
\and
\inferH{VSImp}
  {\always{(\prop \Ra \propB)}}
  {\prop \vs[\emptyset] \propB}
\and
\inferH{VSFrame}
  {\prop \vs[\mask_1][\mask_2] \propB}
  {\prop * \propC \vs[\mask_1 \uplus \mask'][\mask_2 \uplus \mask'] \propB * \propC}
\and
\inferH{VSTimeless}
  {\timeless{\prop}}
  {\later \prop \vs \prop}
\and
\axiomH{InvOpen}
  {\knowInv{\iname}{\prop} \proves \TRUE \vs[\{ \iname \} ][\emptyset] \later \prop}
\and
\axiomH{InvClose}
  {\knowInv{\iname}{\prop} \proves \later \prop \vs[\emptyset][\{ \iname \} ] \TRUE }
\end{mathpar}

\vspace{5pt}
Note that $\timeless{\prop}$ means that $\prop$ does not depend on the step index.
Furthermore, $$\melt \mupd \meltsB \eqdef \always{\All \melt_f. \melt \sep \melt_f \Ra \Exists \meltB \in \meltsB. \meltB \sep \melt_f}$$

\subsection{Derived rules}

\paragraph{Derived structural rules.}
The following are easily derived by unfolding the sugar for Hoare triples and view shifts.
\begin{mathpar}
\inferHB{Disj}
  {\hoare{\prop}{\expr}{\Ret\val.\propC}[\mask] \and \hoare{\propB}{\expr}{\Ret\val.\propC}[\mask]}
  {\hoare{\prop \lor \propB}{\expr}{\Ret\val.\propC}[\mask]}
\and
\inferHB{VSDisj}
  {\prop \vs[\mask_1][\mask_2] \propC \and \propB \vs[\mask_1][\mask_2] \propC}
  {\prop \lor \propB \vs[\mask_1][\mask_2] \propC}
\and
\inferHB{Exist}
  {\All \var. \hoare{\prop}{\expr}{\Ret\val.\propB}[\mask]}
  {\hoare{\Exists \var. \prop}{\expr}{\Ret\val.\propB}[\mask]}
\and
\inferHB{VSExist}
  {\All \var. (\prop \vs[\mask_1][\mask_2] \propB)}
  {(\Exists \var. \prop) \vs[\mask_1][\mask_2] \propB}
\and
\inferHB{BoxOut}
  {\always\propB \provesalways \hoare{\prop}{\expr}{\Ret\val.\propC}[\mask]}
  {\hoare{\prop \land \always{\propB}}{\expr}{\Ret\val.\propC}[\mask]}
\and
\inferHB{VSBoxOut}
  {\always\propB \provesalways \prop \vs[\mask_1][\mask_2] \propC}
  {\prop \land \always{\propB} \vs[\mask_1][\mask_2] \propC}
 \and
\inferH{False}
  {}
  {\hoare{\FALSE}{\expr}{\Ret \val. \prop}[\mask]}
\and
\inferH{VSFalse}
  {}
  {\FALSE \vs[\mask_1][\mask_2] \prop }
\end{mathpar}
The proofs all follow the same pattern, so we only show two of them in detail.
\begin{proof}[Proof of \ruleref{Exist}]
	After unfolding the syntactic sugar for Hoare triples and removing the boxes from premise and conclusion, our goal becomes
	\[
		(\Exists \var. \prop(\var)) \Ra \dynA{\expr}{\Lam\val. \propB}{\mask}
	\]
	(remember that $\var$ is free in $\prop$) and the premise reads
	\[
		\All \var. \prop(\var) \Ra \dynA{\expr}{\Lam\val. \propB}{\mask}.
	\]
	Let $\var$ be given and assume $\prop(\var)$.
	To show $\dynA{\expr}{\Lam\val. \propB}{\mask}$, apply the premise to $\var$ and $\prop(\var)$.
 
	For the other direction, assume
	\[
		\hoare{\Exists \var. \prop(\var)}{\expr}{\Ret\val. \propB}[\mask]
	\]
	and let $\var$ be given.
	We have to show $\hoare{\prop(\var)}{\expr}{\Ret\val. \propB}[\mask]$.
	This trivially follows from \ruleref{Csq} with $\prop(\var) \Ra \Exists \var. \prop(\var)$.
\end{proof}

\begin{proof}[Proof of \ruleref{BoxOut}]
  After unfolding the syntactic sugar for Hoare triples, our goal becomes
  \begin{equation}\label{eq:boxin:goal}
    \always\pfctx \proves \always\bigl(\prop\land\always \propB \Ra \dynA{\expr}{\Lam\val. \propC}{\mask}\bigr)
  \end{equation}
  while our premise reads
  \begin{equation}\label{eq:boxin:as}
    \always\pfctx, \always\propB \proves \always(\prop \Ra \dynA{\expr}{\Lam\val. \propC}{\mask})
  \end{equation}
  By the introduction rules for $\always$ and implication, it suffices to show
  \[  (\always\pfctx), \prop,\always \propB \proves \dynA{\expr}{\Lam\val. \propC}{\mask} \]
  By modus ponens and \ruleref{Necessity}, it suffices to show~\eqref{eq:boxin:as}, which is exactly our assumption.
  
  For the other direction, assume~\eqref{eq:boxin:goal}. We have to show~\eqref{eq:boxin:as}. By \ruleref{AlwaysIntro} and implication introduction, it suffices to show
  \[  (\always\pfctx), \prop,\always \propB \proves \dynA{\expr}{\Lam\val. \propC}{\mask} \]
  which easily follows from~\eqref{eq:boxin:goal}.
\end{proof}

\paragraph{Derived rules for invariants.}
Invariants can be opened around atomic expressions and view shifts.

\begin{mathpar}
\inferH{Inv}
  {\hoare{\later{\propC} * \prop }
          {\expr}
          {\Ret\val. \later{\propC} * \propB }[\mask]
          \and \physatomic{\expr}
  }
  {\knowInv{\iname}{\propC} \proves \hoare{\prop}
          {\expr}
          {\Ret\val. \propB}[\mask \uplus \{ \iname \}]
  }
\and
\inferH{VSInv}
  {\later{\prop} * \propB \vs[\mask_1][\mask_2] \later{\prop} * \propC}
  {\knowInv{\iname}{\prop} \proves \propB \vs[\mask_1 \uplus \{ \iname \}][\mask_2 \uplus \{ \iname \}] \propC}
\end{mathpar}

\begin{proof}[Proof of \ruleref{Inv}]
  Use \ruleref{ACsq} with $\mask_1 \eqdef \mask \cup \{\iname\}$, $\mask_2 \eqdef \mask$.
  The view shifts are obtained by \ruleref{InvOpen} and \ruleref{InvClose} with framing of $\mask$ and $\prop$ or $\propB$, respectively.
\end{proof}

\begin{proof}[Proof of \ruleref{VSInv}]
Analogous to the proof of \ruleref{Inv}, using \ruleref{VSTrans} instead of \ruleref{ACsq}.
\end{proof}

\subsubsection{Unsound rules}

Some rule suggestions (or rather, wishes) keep coming up, which are unsound. We collect them here.
\begin{mathpar}
	\infer
	{P \vs Q}
	{\later P \vs \later Q}
	\and
	\infer
	{\later(P \vs Q)}
	{\later P \vs \later Q}
\end{mathpar}

Of course, the second rule implies the first, so let's focus on that.
Since implications work under $\later$, from $\later P$ we can get $\later \pvs{Q}$.
If we now try to prove $\pvs{\later Q}$, we will be unable to establish world satisfaction in the new world:
We have no choice but to use $\later \pvs{Q}$ at one step index below what we are operating on (because we have it under a $\later$).
We can easily get world satisfaction for that lower step-index (by downwards-closedness of step-indexed predicates).
We can, however, not make much use of the world satisfaction that we get out, becaase it is one step-index too low.

\subsection{Adequacy}

The adequacy statement reads as follows:
\begin{align*}
 &\All \mask, \expr, \val, \pred, i, \state, \state', \tpool'.
 \\&( \proves \hoare{\ownPhys\state}{\expr}{x.\; \pred(x)}[\mask]) \implies
 \\&\cfg{\state}{[i \mapsto \expr]} \step^\ast
     \cfg{\state'}{[i \mapsto \val] \uplus \tpool'} \implies
     \\&\pred(\val)
\end{align*}
where $\pred$ can mention neither resources nor invariants.

\subsection{Axiom lifting}\label{sec:lifting}

The following lemmas help in proving axioms for a particular language.
The first applies to expressions with side-effects, and the second to side-effect-free expressions.
\dave{Update the others, and the example, wrt the new treatment of $\predB$.}
\begin{align*}
 &\All \expr, \state, \pred, \prop, \propB, \mask. \\
 &\textlog{reducible}(e) \implies \\
 &(\All \expr', \state'. \cfg{\state}{\expr} \step \cfg{\state'}{\expr'} \implies \pred(\expr', \state')) \implies \\
 &{} \proves \bigl( (\All \expr', \state'. \pred (\expr', \state') \Ra \hoare{\prop}{\expr'}{\Ret\val. \propB}[\mask]) \Ra \hoare{ \later \prop * \ownPhys{\state} }{\expr}{\Ret\val. \propB}[\mask] \bigr) \\
 \quad\\
 &\All \expr, \pred, \prop, \propB, \mask. \\
 &\textlog{reducible}(e) \implies \\
 &(\All \state, \expr_2, \state_2. \cfg{\state}{\expr} \step \cfg{\state_2}{\expr_2} \implies \state_2 = \state \land \pred(\expr_2)) \implies \\
 &{} \proves \bigl( (\All \expr'. \pred(\expr') \Ra \hoare{\prop}{\expr'}{\Ret\val. \propB}[\mask]) \Ra \hoare{\later\prop}{\expr}{\Ret\val. \propB}[\mask] \bigr)
\end{align*}
Note that $\pred$ is a meta-logic predicate---it does not depend on any world or resources being owned.

The following specializations cover all cases of a heap-manipulating lambda calculus like $F_{\mu!}$.
\begin{align*}
 &\All \expr, \expr', \prop, \propB, \mask. \\
 &\textlog{reducible}(e) \implies \\
 &(\All \state, \expr_2, \state_2. \cfg{\state}{\expr} \step \cfg{\state_2}{\expr_2} \implies \state_2 = \state \land \expr_2 = \expr') \implies \\
 &{} \proves (\hoare{\prop}{\expr'}{\Ret\val. \propB}[\mask] \Ra \hoare{\later\prop}{\expr}{\Ret\val. \propB}[\mask] ) \\
 \quad \\
 &\All \expr, \state, \pred, \mask. \\
 &\textlog{atomic}(e) \implies \\
 &\bigl(\All \expr_2, \state_2. \cfg{\state}{\expr} \step \cfg{\state_2}{\expr_2} \implies \pred(\expr_2, \state_2)\bigr) \implies \\
 &{} \proves (\hoare{ \ownPhys{\state} }{\expr}{\Ret\val. \Exists\state'. \ownPhys{\state'} \land \pred(\val, \state') }[\mask] )
\end{align*}
The first is restricted to deterministic pure reductions, like $\beta$-reduction.
The second is suited to proving triples for (possibly non-deterministic) atomic expressions; for example, with $\expr \eqdef \;!\ell$ (dereferencing $\ell$) and $\state \eqdef h \mtimes \ell \mapsto \valB$ and $\pred(\val, \state') \eqdef \state' = (h \mtimes \ell \mapsto \valB) \land \val = \valB$, one obtains the axiom $\All h, \ell, \valB. \hoare{\ownPhys{h \mtimes \ell \mapsto \valB}}{!\ell}{\Ret\val. \val = \valB \land \ownPhys{h \mtimes \ell \mapsto \valB} }$.
%Axioms for CAS-like operations can be obtained by first deriving rules for the two possible cases, and then using the disjunction rule.


\section{Derived constructions}

In this section we describe some constructions that we will use throughout the rest of the appendix.

\subsection{Global monoid}

Hereinafter we assume the global monoid (served up as a parameter to Iris) is obtained from a family of monoids $(M_i)_{i \in I}$ by first applying the construction for finite partial functions to each~(\Sref{sec:fpfunm}), and then applying the product construction~(\Sref{sec:prodm}):
\[ M \eqdef \prod_{i \in I} \textdom{GhName} \fpfn M_i \]
We don't care so much about what concretely $\textdom{GhName}$ is, as long as it is countable and infinite.
We write $\ownGhost{\gname}{\melt : M_i}$ (or just $\ownGhost{\gname}{\melt}$ if $M_i$ is clear from the context) for $\ownGGhost{[i \mapsto [\gname \mapsto \melt]]}$ when $\melt \in \mcarp {M_i}$, and for $\FALSE$ when $\melt = \mzero_{M_i}$.
In other words, $\ownGhost{\gname}{\melt : M_i}$ asserts that in the current state of monoid $M_i$, the name $\gname$ is allocated and has at least value $\melt$.

From~\ruleref{FpUpd} and the multiplications and frame-preserving updates in~\Sref{sec:prodm} and~\Sref{sec:fpfunm}, we have the following derived rules.
\begin{mathpar}
	\axiomH{NewGhost}{
		\TRUE \vs \Exists\gname. \ownGhost\gname{\melt : M_i}
	}
	\and
	\inferH{GhostUpd}
    {\melt \mupd_{M_i} B}
    {\ownGhost\gname{\melt : M_i} \vs \Exists \meltB\in B. \ownGhost\gname{\meltB : M_i}}
  \and
  \axiomH{GhostEq}
    {\ownGhost\gname{\melt : M_i} * \ownGhost\gname{\meltB : M_i} \Lra \ownGhost\gname{\melt\mtimes\meltB : M_i}}

  \axiomH{GhostUnit}
    {\TRUE \Ra \ownGhost{\gname}{\munit : M_i}}

  \axiomH{GhostZero}
    {\ownGhost\gname{\mzero : M_i} \Ra \FALSE}

  \axiomH{GhostTimeless}
    {\timeless{\ownGhost\gname{\melt : M_i}}}
\end{mathpar}

\subsection{STSs with interpretation}\label{sec:stsinterp}

Building on \Sref{sec:stsmon}, after constructing the monoid $\STSMon{\STSS}$ for a particular STS, we can use an invariant to tie an interpretation, $\pred : \STSS \to \Prop$, to the STS's current state, recovering CaReSL-style reasoning~\cite{caresl}.

An STS invariant asserts authoritative ownership of an STS's current state and that state's interpretation:
\begin{align*}
  \STSInv(\STSS, \pred, \gname) \eqdef{}& \Exists s \in \STSS. \ownGhost{\gname}{(s, \STSS, \emptyset):\STSMon{\STSS}} * \pred(s) \\
  \STS(\STSS, \pred, \gname, \iname) \eqdef{}& \knowInv{\iname}{\STSInv(\STSS, \pred, \gname)}
\end{align*}

We can specialize \ruleref{NewInv}, \ruleref{InvOpen}, and \ruleref{InvClose} to STS invariants:
\begin{mathpar}
 \inferH{NewSts}
  {\infinite(\mask)}
  {\later\pred(s) \vs[\mask] \Exists \iname \in \mask, \gname.   \STS(\STSS, \pred, \gname, \iname) * \ownGhost{\gname}{(s, \STST \setminus \STSL(s)) : \STSMon{\STSS}}}
 \and
 \axiomH{StsOpen}
  {  \STS(\STSS, \pred, \gname, \iname) \vdash \ownGhost{\gname}{(s_0, T) : \STSMon{\STSS}} \vsE[\{\iname\}][\emptyset] \Exists s\in \upclose(\{s_0\}, T). \later\pred(s) * \ownGhost{\gname}{(s, \upclose(\{s_0\}, T), T):\STSMon{\STSS}}}
 \and
 \axiomH{StsClose}
  {  \STS(\STSS, \pred, \gname, \iname), (s, T) \ststrans (s', T')  \proves \later\pred(s') * \ownGhost{\gname}{(s, S, T):\STSMon{\STSS}} \vs[\emptyset][\{\iname\}] \ownGhost{\gname}{(s', T') : \STSMon{\STSS}} }
\end{mathpar}
\begin{proof}
\ruleref{NewSts} uses \ruleref{NewGhost} to allocate $\ownGhost{\gname}{(s, \upclose(s, T), T) : \STSMon{\STSS}}$ where $T \eqdef \STST \setminus \STSL(s)$, and \ruleref{NewInv}.

\ruleref{StsOpen} just uses \ruleref{InvOpen} and \ruleref{InvClose} on $\iname$, and the monoid equality $(s, \upclose(\{s_0\}, T), T) = (s, \STSS, \emptyset) \mtimes (\munit, \upclose(\{s_0\}, T), T)$.

\ruleref{StsClose} applies \ruleref{StsStep} and \ruleref{InvClose}.
\end{proof}

Using these view shifts, we can prove STS variants of the invariant rules \ruleref{Inv} and \ruleref{VSInv}~(compare the former to CaReSL's island update rule~\cite{caresl}):
\begin{mathpar}
 \inferH{Sts}
  {\All s \in \upclose(\{s_0\}, T). \hoare{\later\pred(s) * P}{\expr}{\Ret \val. \Exists s', T'. (s, T) \ststrans (s', T') * \later\pred(s') * Q}[\mask]
   \and \physatomic{\expr}}
  {  \STS(\STSS, \pred, \gname, \iname) \vdash \hoare{\ownGhost{\gname}{(s_0, T):\STSMon{\STSS}} * P}{\expr}{\Ret \val. \Exists s', T'. \ownGhost{\gname}{(s', T'):\STSMon{\STSS}} * Q}[\mask \uplus \{\iname\}]}
 \and
 \inferH{VSSts}
  {\forall s \in \upclose(\{s_0\}, T).\; \later\pred(s) * P \vs[\mask_1][\mask_2] \exists s', T'.\; (s, T) \ststrans (s', T') * \later\pred(s') * Q}
  {  \STS(\STSS, \pred, \gname, \iname) \vdash \ownGhost{\gname}{(s_0, T):\STSMon{\STSS}} * P \vs[\mask_1 \uplus \{\iname\}][\mask_2 \uplus \{\iname\}] \Exists s', T'. \ownGhost{\gname}{(s', T'):\STSMon{\STSS}} * Q}
\end{mathpar}

\begin{proof}[Proof of \ruleref{Sts}]\label{pf:sts}
 We have to show
 \[\hoare{\ownGhost{\gname}{(s_0, T):\STSMon{\STSS}} * P}{\expr}{\Ret \val. \Exists s', T'. \ownGhost{\gname}{(s', T'):\STSMon{\STSS}} * Q}[\mask \uplus \{\iname\}]\]
 where $\val$, $s'$, $T'$ are free in $Q$.
 
 First, by \ruleref{ACsq} with \ruleref{StsOpen} and \ruleref{StsClose} (after moving $(s, T) \ststrans (s', T')$ into the view shift using \ruleref{VSBoxOut}), it suffices to show
 \[\hoareV{\Exists s\in \upclose(\{s_0\}, T). \later\pred(s) * \ownGhost{\gname}{(s, \upclose(\{s_0\}, T), T)} * P}{\expr}{\Ret \val. \Exists s, T, S, s', T'. (s, T) \ststrans (s', T') * \later\pred(s') * \ownGhost{\gname}{(s, S, T):\STSMon{\STSS}} * Q(\val, s', T')}[\mask]\]

 Now, use \ruleref{Exist} to move the $s$ from the precondition into the context and use \ruleref{Csq} to (i)~fix the $s$ and $T$ in the postcondition to be the same as in the precondition, and (ii)~fix $S \eqdef \upclose(\{s_0\}, T)$.
 It remains to show:
 \[\hoareV{s\in \upclose(\{s_0\}, T) * \later\pred(s) * \ownGhost{\gname}{(s, \upclose(\{s_0\}, T), T)} * P}{\expr}{\Ret \val. \Exists s', T'. (s, T) \ststrans (s', T') * \later\pred(s') * \ownGhost{\gname}{(s, \upclose(\{s_0\}, T), T)} * Q(\val, s', T')}[\mask]\]
 
 Finally, use \ruleref{BoxOut} to move $s\in \upclose(\{s_0\}, T)$ into the context, and \ruleref{Frame} on $\ownGhost{\gname}{(s, \upclose(\{s_0\}, T), T)}$:
 \[s\in \upclose(\{s_0\}, T) \vdash \hoare{\later\pred(s) * P}{\expr}{\Ret \val. \Exists s', T'. (s, T) \ststrans (s', T') * \later\pred(s') * Q(\val, s', T')}[\mask]\]
 
 This holds by our premise.
\end{proof}

% \begin{proof}[Proof of \ruleref{VSSts}]
% This is similar to above, so we only give the proof in short notation:

% \hproof{%
% 	Context: $\knowInv\iname{\STSInv(\STSS, \pred, \gname)}$ \\
% 	\pline[\mask_1 \uplus \{\iname\}]{
% 		\ownGhost\gname{(s_0, T)} * P
% 	} \\
% 	\pline[\mask_1]{%
% 		\Exists s. \later\pred(s) * \ownGhost\gname{(s, S, T)} * P
% 	} \qquad by \ruleref{StsOpen} \\
% 	Context: $s \in S \eqdef \upclose(\{s_0\}, T)$ \\
% 	\pline[\mask_2]{%
% 		 \Exists s', T'. \later\pred(s') * Q(s', T') * \ownGhost\gname{(s, S, T)}
% 	} \qquad by premiss \\
% 	Context: $(s, T) \ststrans (s', T')$ \\
% 	\pline[\mask_2 \uplus \{\iname\}]{
% 		\ownGhost\gname{(s', T')} * Q(s', T')
% 	} \qquad by \ruleref{StsClose}
% }
% \end{proof}

\subsection{Authoritative monoids with interpretation}\label{sec:authinterp}

Building on \Sref{sec:auth}, after constructing the monoid $\auth{M}$ for a cancellative monoid $M$, we can tie an interpretation, $\pred : \mcarp{M} \to \Prop$, to the authoritative element of $M$, recovering reasoning that is close to the sharing rule in~\cite{krishnaswami+:icfp12}.

Let $\pred_\bot$ be the extension of $\pred$ to $\mcar{M}$ with $\pred_\bot(\mzero) = \FALSE$.
Now define
\begin{align*}
  \AuthInv(M, \pred, \gname) \eqdef{}& \exists \melt \in \mcar{M}.\; \ownGhost{\gname}{\authfull \melt:\auth{M}} * \pred_\bot(\melt) \\
  \Auth(M, \pred, \gname, \iname) \eqdef{}& M~\textlog{cancellative} \land \knowInv{\iname}{\AuthInv(M, \pred, \gname)}
\end{align*}

The frame-preserving updates for $\auth{M}$ gives rise to the following view shifts:
\begin{mathpar}
 \inferH{NewAuth}
  {\infinite(\mask) \and M~\textlog{cancellative}}
  {\later\pred_\bot(a) \vs[\mask] \exists \iname \in \mask, \gname.\; \Auth(M, \pred, \gname, \iname) * \ownGhost{\gname}{\authfrag a : \auth{M}}}
 \and
 \axiomH{AuthOpen}
  {\Auth(M, \pred, \gname, \iname) \vdash \ownGhost{\gname}{\authfrag \melt : \auth{M}} \vsE[\{\iname\}][\emptyset] \exists \melt_f.\; \later\pred_\bot(\melt \mtimes \melt_f) * \ownGhost{\gname}{\authfull \melt \mtimes \melt_f, \authfrag a:\auth{M}}}
 \and
 \axiomH{AuthClose}
  {\Auth(M, \pred, \gname, \iname) \vdash \later\pred_\bot(\meltB \mtimes \melt_f) * \ownGhost{\gname}{\authfull a \mtimes \melt_f, \authfrag a:\auth{M}} \vs[\emptyset][\{\iname\}] \ownGhost{\gname}{\authfrag \meltB : \auth{M}} }
\end{mathpar}

These view shifts in turn can be used to prove variants of the invariant rules:
\begin{mathpar}
 \inferH{Auth}
  {\forall \melt_f.\; \hoare{\later\pred_\bot(a \mtimes \melt_f) * P}{\expr}{\Ret\val. \exists \meltB.\; \later\pred_\bot(\meltB\mtimes \melt_f) * Q}[\mask]
   \and \physatomic{\expr}}
  {\Auth(M, \pred, \gname, \iname) \vdash \hoare{\ownGhost{\gname}{\authfrag a:\auth{M}} * P}{\expr}{\Ret\val. \exists \meltB.\; \ownGhost{\gname}{\authfrag \meltB:\auth{M}} * Q}[\mask \uplus \{\iname\}]}
 \and
 \inferH{VSAuth}
  {\forall \melt_f.\; \later\pred_\bot(a \mtimes \melt_f) * P \vs[\mask_1][\mask_2] \exists \meltB.\; \later\pred_\bot(\meltB \mtimes \melt_f) * Q(\meltB)}
  {\Auth(M, \pred, \gname, \iname) \vdash
   \ownGhost{\gname}{\authfrag a:\auth{M}} * P \vs[\mask_1 \uplus \{\iname\}][\mask_2 \uplus \{\iname\}]
   \exists \meltB.\; \ownGhost{\gname}{\authfrag \meltB:\auth{M}} * Q(\meltB)}
\end{mathpar}


\subsection{Ghost heap}
\label{sec:ghostheap}%

We define a simple ghost heap with fractional permissions.
Some modules require a few ghost names per module instance to properly manage ghost state, but would like to expose to clients a single logical name (avoiding clutter).
In such cases we use these ghost heaps.

We seek to implement the following interface:
\newcommand{\GRefspecmaps}{\textsf{GMapsTo}}%
\begin{align*}
 \exists& {\fgmapsto[]} : \textsort{Val} \times \mathbb{Q}_{>} \times \textsort{Val} \ra \textsort{Prop}.\;\\
  & \All x, q, v. x \fgmapsto[q] v \Ra x \fgmapsto[q] v \land q \in (0, 1] \\
  &\forall x, q_1, q_2, v, w.\; x \fgmapsto[q_1] v * x \fgmapsto[q_2] w \Leftrightarrow x \fgmapsto[q_1 + q_2] v * v = w\\
  & \forall v.\; \TRUE \vs[\emptyset] \exists x.\; x \fgmapsto[1] v \\
  & \forall x, v, w.\; x \fgmapsto[1] v \vs[\emptyset] x \fgmapsto[1] w
\end{align*}
We write $x \fgmapsto v$ for $\exists q.\; x \fgmapsto[q] v$ and $x \gmapsto v$ for $x \fgmapsto[1] v$.
Note that $x \fgmapsto v$ is duplicable but cannot be boxed (as it depends on resources); \ie we have $x \fgmapsto v \Lra x \fgmapsto v * x \fgmapsto v$ but not $x \fgmapsto v \Ra \always x \fgmapsto v$.

To implement this interface, allocate an instance $\gname_G$ of $\FHeap(\textdom{Val})$ and define
\[
	x \fgmapsto[q] v \eqdef
	  \begin{cases}
    	\ownGhost{\gname_G}{x \mapsto (q, v)} & \text{if $q \in (0, 1]$} \\
    	\FALSE & \text{otherwise}
    \end{cases}
\]
The view shifts in the specification follow immediately from \ruleref{GhostUpd} and the frame-preserving updates in~\Sref{sec:fheapm}.
The first implication is immediate from the definition.
The second implication follows by case distinction on $q_1 + q_2 \in (0, 1]$.


%%% Local Variables:
%%% mode: latex
%%% TeX-master: "iris"
%%% End:
