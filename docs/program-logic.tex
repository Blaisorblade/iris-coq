\let\bar\overline

\section{Language}
\label{sec:language}

A \emph{language} $\Lang$ consists of a set \textdom{Expr} of \emph{expressions} (metavariable $\expr$), a set \textdom{Val} of \emph{values} (metavariable $\val$), and a set \textdom{State} of \emph{states} (metvariable $\state$) such that
\begin{itemize}
\item There exist functions $\ofval : \textdom{Val} \to \textdom{Expr}$ and $\toval : \textdom{Expr} \pfn \textdom{val}$ (notice the latter is partial), such that
\begin{mathpar} {\All \expr, \val. \toval(\expr) = \val \Ra \ofval(\val) = \expr} \and {\All\val. \toval(\ofval(\val)) = \val} 
\end{mathpar}
\item There exists a \emph{primitive reduction relation} \[(-,- \step -,-,-) \subseteq \textdom{Expr} \times \textdom{State} \times \textdom{Expr} \times \textdom{State} \times (\cup_n \textdom{Expr}^n)\]
  A reduction $\expr_1, \state_1 \step \expr_2, \state_2, \overline\expr$ indicates that, when $\expr_1$ reduces to $\expr_2$, the new threads in the list $\overline\expr$ is forked off.
  We will write $\expr_1, \state_1 \step \expr_2, \state_2$ for $\expr_1, \state_1 \step \expr_2, \state_2, ()$, \ie when no threads are forked off. \\
\item All values are stuck:
\[ \expr, \_ \step  \_, \_, \_ \Ra \toval(\expr) = \bot \]
\end{itemize}

\begin{defn}
  An expression $\expr$ and state $\state$ are \emph{reducible} (written $\red(\expr, \state)$) if
  \[ \Exists \expr_2, \state_2, \bar\expr. \expr,\state \step \expr_2,\state_2,\bar\expr \]
\end{defn}

\begin{defn}
  An expression $\expr$ is said to be \emph{atomic} if it reduces in one step to a value:
  \[ \All\state_1, \expr_2, \state_2, \bar\expr. \expr, \state_1 \step \expr_2, \state_2, \bar\expr \Ra \Exists \val_2. \toval(\expr_2) = \val_2 \]
\end{defn}

\begin{defn}[Context]
  A function $\lctx : \textdom{Expr} \to \textdom{Expr}$ is a \emph{context} if the following conditions are satisfied:
  \begin{enumerate}[itemsep=0pt]
  \item $\lctx$ does not turn non-values into values:\\
    $\All\expr. \toval(\expr) = \bot \Ra \toval(\lctx(\expr)) = \bot $
  \item One can perform reductions below $\lctx$:\\
    $\All \expr_1, \state_1, \expr_2, \state_2, \bar\expr. \expr_1, \state_1 \step \expr_2,\state_2,\bar\expr \Ra \lctx(\expr_1), \state_1 \step \lctx(\expr_2),\state_2,\bar\expr $
  \item Reductions stay below $\lctx$ until there is a value in the hole:\\
    $\All \expr_1', \state_1, \expr_2, \state_2, \bar\expr. \toval(\expr_1') = \bot \land \lctx(\expr_1'), \state_1 \step \expr_2,\state_2,\bar\expr \Ra \Exists\expr_2'. \expr_2 = \lctx(\expr_2') \land \expr_1', \state_1 \step \expr_2',\state_2,\bar\expr $
  \end{enumerate}
\end{defn}

\subsection{Concurrent language}

For any language $\Lang$, we define the corresponding thread-pool semantics.

\paragraph{Machine syntax}
\[
	\tpool \in \textdom{ThreadPool} \eqdef \bigcup_n \textdom{Expr}^n
\]

\judgment[Machine reduction]{\cfg{\tpool}{\state} \step
  \cfg{\tpool'}{\state'}}
\begin{mathpar}
\infer
  {\expr_1, \state_1 \step \expr_2, \state_2, \bar\expr}
  {\cfg{\tpool \dplus [\expr_1] \dplus \tpool'}{\state_1} \step
     \cfg{\tpool \dplus [\expr_2] \dplus \tpool' \dplus \bar\expr}{\state_2}}
\end{mathpar}

\clearpage
\section{Program Logic}
\label{sec:program-logic}

This section describes how to build a program logic for an arbitrary language (\cf \Sref{sec:language}) on top of the logic described in \Sref{sec:dc-logic}.
So in the following, we assume that some language $\Lang$ was fixed.

\subsection{World Satisfaction, Invariants, View Shifts}
\label{sec:invariants}

To introduce invariants into our logic, we will define weakest precondition to explicitly thread through the proof that all the invariants are maintained throughout program execution.
However, in order to be able to access invariants, we will also have to provide a way to \emph{temporarily disable} (or ``open'') them.
To this end, we use tokens that manage which invariants are currently enabled.

We assume to have the following four CMRAs available:
\begin{align*}
  \textmon{State} \eqdef{}& \authm(\exm(\textdom{State})) \\
  \textmon{Inv} \eqdef{}& \authm(\mathbb N \fpfn \agm(\latert \iPreProp)) \\
  \textmon{En} \eqdef{}& \pset{\mathbb N} \\
  \textmon{Dis} \eqdef{}& \finpset{\mathbb N}
\end{align*}
The last two are the tokens used for managing invariants, $\textmon{Inv}$ is the monoid used to manage the invariants themselves.
Finally, $\textmon{State}$ is used to provide the program with a view of the physical state of the machine.

Furthermore, we assume that instances named $\gname_{\textmon{State}}$, $\gname_{\textmon{Inv}}$, $\gname_{\textmon{En}}$ and $\gname_{\textmon{Dis}}$ of these CMRAs have been created.
(We will discuss later how this assumption is discharged.)

\paragraph{World Satisfaction.}
We can now define the assertion $W$ (\emph{world satisfaction}) which ensures that the enabled invariants are actually maintained:
\begin{align*}
  W \eqdef{}& \Exists I : \mathbb N \fpfn \Prop. \ownGhost{\gname_{\textmon{Inv}}}{\authfull \setComp{\iname \mapsto \aginj(\latertinj(\wIso(I(\iname))))}{\iname \in \dom(I)}} * \Sep_{\iname \in \dom(I)} \left( \later I(\iname) * \ownGhost{\gname_{\textmon{Dis}}}{\set{\iname}} \lor \ownGhost{\gname_{\textmon{En}}}{\set{\iname}} \right)
\end{align*}

\paragraph{Invariants.}
The following assertion states that an invariant with name $\iname$ exists and maintains assertion $\prop$:
\[ \knowInv\iname\prop \eqdef \ownGhost{\gname_{\textmon{Inv}}}{\authfrag\set{\iname \mapsto  \aginj(\latertinj(\wIso(\prop)))}} \]

\paragraph{View Updates and View Shifts.}
Next, we define \emph{view updates}, which are essentially the same as the resource updates of the base logic ($\Sref{sec:base-logic}$), except that they also have access to world satisfaction and can enable and disable invariants:
\[ \pvs[\mask_1][\mask_2] \prop \eqdef W * \ownGhost{\gname_{\textmon{En}}}{\mask_1} \wand \upd\diamond (W * \ownGhost{\gname_{\textmon{En}}}{\mask_2} * \prop) \]
Here, $\mask_1$ and $\mask_2$ are the \emph{masks} of the view update, defining which invariants have to be (at least!) available before and after the update.
We use $\top$ as symbol for the largest possible mask, $\mathbb N$.
We will write $\pvs[\mask] \prop$ for $\pvs[\mask][\mask]\prop$.
%
View updates satisfy the following basic proof rules:
\begin{mathpar}
\infer[vup-mono]
{\prop \proves \propB}
{\pvs[\mask_1][\mask_2] \prop \proves \pvs[\mask_1][\mask_2] \propB}

\infer[vup-intro-mask]
{\mask_2 \subseteq \mask_1}
{\prop \proves \pvs[\mask_1][\mask_2]\pvs[\mask_2][\mask_1] \prop}

\infer[vup-trans]
{}
{\pvs[\mask_1][\mask_2] \pvs[\mask_2][\mask_3] \prop \proves \pvs[\mask_1][\mask_3] \prop}

\infer[vup-upd]
{}{\upd\prop \proves \pvs[\mask] \prop}

\infer[vup-frame]
{}{\propB * \pvs[\mask_1][\mask_2]\prop \proves \pvs[\mask_1 \uplus \mask_\f][\mask_2 \uplus \mask_\f] \propB * \prop}

\inferH{vup-update}
{\melt \mupd \meltsB}
{\ownM\melt \proves \pvs[\mask] \Exists\meltB\in\meltsB. \ownM\meltB}

\infer[vup-timeless]
{\timeless\prop}
{\later\prop \proves \pvs[\mask] \prop}
%
% \inferH{vup-allocI}
% {\text{$\mask$ is infinite}}
% {\later\prop \proves \pvs[\mask] \Exists \iname \in \mask. \knowInv\iname\prop}
%gov
% \inferH{vup-openI}
% {}{\knowInv\iname\prop \proves \pvs[\set\iname][\emptyset] \later\prop}
%
% \inferH{vup-closeI}
% {}{\knowInv\iname\prop \land \later\prop \proves \pvs[\emptyset][\set\iname] \TRUE}
\end{mathpar}
(There are no rules related to invariants here. Those rules will be discussed later, in \Sref{sec:invariants}.)

We further define the notions of \emph{view shifts} and \emph{linear view shifts}:
\begin{align*}
  \prop \vs[\mask_1][\mask_2] \propB \eqdef{}& \always(\prop \Ra \pvs[\mask_1][\mask_2] \propB) \\
  \prop \vsW[\mask_1][\mask_2] \propB \eqdef{}& \prop \wand \pvs[\mask_1][\mask_2] \propB
\end{align*}
These two are useful when writing down specifications, but for reasoning, it is typically easier to just work directly with view updates.
Still, just to give an idea of what view shifts ``are'', here are some proof rules for them:
\begin{mathparpagebreakable}
\inferH{vs-update}
  {\melt \mupd \meltsB}
  {\ownGhost\gname{\melt} \vs \exists \meltB \in \meltsB.\; \ownGhost\gname{\meltB}}
\and
\inferH{vs-trans}
  {\prop \vs[\mask_1][\mask_2] \propB \and \propB \vs[\mask_2][\mask_3] \propC}
  {\prop \vs[\mask_1][\mask_3] \propC}
\and
\inferH{vs-imp}
  {\always{(\prop \Ra \propB)}}
  {\prop \vs[\emptyset] \propB}
\and
\inferH{vs-mask-frame}
  {\prop \vs[\mask_1][\mask_2] \propB}
  {\prop \vs[\mask_1 \uplus \mask'][\mask_2 \uplus \mask'] \propB}
\and
\inferH{vs-frame}
  {\prop \vs[\mask_1][\mask_2] \propB}
  {\prop * \propC \vs[\mask_1][\mask_2] \propB * \propC}
\and
\inferH{vs-timeless}
  {\timeless{\prop}}
  {\later \prop \vs \prop}

% \inferH{vs-allocI}
%   {\infinite(\mask)}
%   {\later{\prop} \vs[\mask] \exists \iname\in\mask.\; \knowInv{\iname}{\prop}}
% \and
% \axiomH{vs-openI}
%   {\knowInv{\iname}{\prop} \proves \TRUE \vs[\{ \iname \} ][\emptyset] \later \prop}
% \and
% \axiomH{vs-closeI}
%   {\knowInv{\iname}{\prop} \proves \later \prop \vs[\emptyset][\{ \iname \} ] \TRUE }
%
\inferHB{vs-disj}
  {\prop \vs[\mask_1][\mask_2] \propC \and \propB \vs[\mask_1][\mask_2] \propC}
  {\prop \lor \propB \vs[\mask_1][\mask_2] \propC}
\and
\inferHB{vs-exist}
  {\All \var. (\prop \vs[\mask_1][\mask_2] \propB)}
  {(\Exists \var. \prop) \vs[\mask_1][\mask_2] \propB}
\and
\inferHB{vs-always}
  {\always\propB \proves \prop \vs[\mask_1][\mask_2] \propC}
  {\prop \land \always{\propB} \vs[\mask_1][\mask_2] \propC}
 \and
\inferH{vs-false}
  {}
  {\FALSE \vs[\mask_1][\mask_2] \prop }
\end{mathparpagebreakable}

\subsection{Weakest Precondition}

Finally, we can define the core piece of the program logic, the assertion that reasons about program behavior: Weakest precondition, from which Hoare triples will be derived.

\paragraph{Defining weakest precondition.}
We assume that everything making up the definition of the language, \ie values, expressions, states, the conversion functions, reduction relation and all their properties, are suitably reflected into the logic (\ie they are part of the signature $\Sig$).

\begin{align*}
  \textdom{wp} \eqdef{}& \MU \textdom{wp}. \Lam \mask, \expr, \pred. \\
        & (\Exists\val. \toval(\expr) = \val \land \pvs[\mask] \prop) \lor {}\\
        & \Bigl(\toval(\expr) = \bot \land \All \state. \ownGhost{\gname_{\textmon{State}}}{\authfull \state} \vsW[\mask][\emptyset] {}\\
        &\qquad \red(\expr, \state) * \later\All \expr', \state', \bar\expr. (\expr, \state \step \expr', \state', \bar\expr) \vsW[\emptyset][\mask] {}\\
        &\qquad\qquad \ownGhost{\gname_{\textmon{State}}}{\authfull \state'} * \textdom{wp}(\mask, \expr', \pred) * \Sep_{\expr'' \in \bar\expr} \textdom{wp}(\top, \expr'', \Lam \any. \TRUE)\Bigr) \\
%  (* value case *)
  \wpre\expr[\mask]{\Ret\val. \prop} \eqdef{}& \textdom{wp}(\mask, \expr, \Lam\val.\prop)
\end{align*}
If we leave away the mask, we assume it to default to $\top$.

This ties the authoritative part of \textmon{State} to the actual physical state of the reduction witnessed by the weakest precondition.
The fragment will then be available to the user of the logic, as their way of talking about the physical state:
\[ \ownPhys\state \eqdef \ownGhost{\gname_{\textmon{State}}}{\authfrag \state} \]

\paragraph{Laws of weakest precondition.}
The following rules can all be derived inside the DC logic:
\begin{mathpar}
\infer[wp-value]
{}{\prop[\val/\var] \proves \wpre{\val}[\mask]{\Ret\var.\prop}}

\infer[wp-mono]
{\mask_1 \subseteq \mask_2 \and \vctx,\var:\textlog{val}\mid\prop \proves \propB}
{\vctx\mid\wpre\expr[\mask_1]{\Ret\var.\prop} \proves \wpre\expr[\mask_2]{\Ret\var.\propB}}

\infer[pvs-wp]
{}{\pvs[\mask] \wpre\expr[\mask]{\Ret\var.\prop} \proves \wpre\expr[\mask]{\Ret\var.\prop}}

\infer[wp-pvs]
{}{\wpre\expr[\mask]{\Ret\var.\pvs[\mask] \prop} \proves \wpre\expr[\mask]{\Ret\var.\prop}}

\infer[wp-atomic]
{\physatomic{\expr}}
{\pvs[\mask_1][\mask_2] \wpre\expr[\mask_2]{\Ret\var. \pvs[\mask_2][\mask_1]\prop}
 \proves \wpre\expr[\mask_1]{\Ret\var.\prop}}

\infer[wp-frame]
{}{\propB * \wpre\expr[\mask]{\Ret\var.\prop} \proves \wpre\expr[\mask]{\Ret\var.\propB*\prop}}

\infer[wp-frame-step]
{\toval(\expr) = \bot \and \mask_2 \subseteq \mask_1}
{\wpre\expr[\mask_2]{\Ret\var.\prop} * \pvs[\mask_1][\mask_2]\later\pvs[\mask_2][\mask_1]\propB \proves \wpre\expr[\mask_1]{\Ret\var.\propB*\prop}}

\infer[wp-bind]
{\text{$\lctx$ is a context}}
{\wpre\expr[\mask]{\Ret\var. \wpre{\lctx(\ofval(\var))}[\mask]{\Ret\varB.\prop}} \proves \wpre{\lctx(\expr)}[\mask]{\Ret\varB.\prop}}
\end{mathpar}

We will also want rules that connect weakest preconditions to the operational semantics of the language.
In order to cover the most general case, those rules end up being more complicated:
\begin{mathpar}
  \infer[wp-lift-step]
  {\toval(\expr_1) = \bot}
  { {\begin{inbox} % for some crazy reason, LaTeX is actually sensitive to the space between the "{ {" here and the "} }" below...
        ~~\pvs[\mask][\emptyset] \Exists \state_1. \red(\expr_1,\state_1) * \later\ownPhys{\state_1} * {}\\\qquad~~ \later\All \expr_2, \state_2, \bar\expr. \Bigl( (\expr_1, \state_1 \step \expr_2, \state_2, \bar\expr) * \ownPhys{\state_2} \Bigr) \wand \pvs[\emptyset][\mask] \Bigl(\wpre{\expr_2}[\mask]{\Ret\var.\prop} * \Sep_{\expr_\f \in \bar\expr} \wpre{\expr_\f}[\top]{\Ret\any.\TRUE}\Bigr)  {}\\\proves \wpre{\expr_1}[\mask]{\Ret\var.\prop}
      \end{inbox}} }
\\\\
  \infer[wp-lift-pure-step]
  {\toval(\expr_1) = \bot \and
   \All \state_1. \red(\expr_1, \state_1) \and
   \All \state_1, \expr_2, \state_2, \bar\expr. \expr_1,\state_1 \step \expr_2,\state_2,\bar\expr \Ra \state_1 = \state_2 }
  {\later\All \state, \expr_2, \bar\expr. (\expr_1,\state \step \expr_2, \state,\bar\expr)  \Ra \wpre{\expr_2}[\mask]{\Ret\var.\prop} * \Sep_{\expr_\f \in \bar\expr} \wpre{\expr_\f}[\top]{\Ret\any.\TRUE} \proves \wpre{\expr_1}[\mask]{\Ret\var.\prop}}
\end{mathpar}

We can further derive some slightly simpler rules for special cases:
We can derive some specialized forms of the lifting axioms for the operational semantics.
\begin{mathparpagebreakable}
  \infer[wp-lift-atomic-step]
  {\atomic(\expr_1) \and
   \red(\expr_1, \state_1)}
  { {\begin{inbox}~~\later\ownPhys{\state_1} * \later\All \val_2, \state_2, \bar\expr. (\expr_1,\state_1 \step \ofval(\val),\state_2,\bar\expr)  * \ownPhys{\state_2} \wand \prop[\val_2/\var] * \Sep_{\expr_\f \in \bar\expr} \wpre{\expr_\f}[\top]{\Ret\any.\TRUE} {}\\ \proves  \wpre{\expr_1}[\mask_1]{\Ret\var.\prop}
  \end{inbox}} }

  \infer[wp-lift-atomic-det-step]
  {\atomic(\expr_1) \and
   \red(\expr_1, \state_1) \and
   \All \expr'_2, \state'_2, \bar\expr'. \expr_1,\state_1 \step \expr'_2,\state'_2,\bar\expr' \Ra \state_2 = \state_2' \land \toval(\expr_2') = \val_2 \land \bar\expr = \bar\expr'}
  {\later\ownPhys{\state_1} * \later \Bigl(\ownPhys{\state_2} \wand \prop[\val_2/\var] * \Sep_{\expr_\f \in \bar\expr} \wpre{\expr_\f}[\top]{\Ret\any.\TRUE} \Bigr) \proves \wpre{\expr_1}[\mask_1]{\Ret\var.\prop}}

  \infer[wp-lift-pure-det-step]
  {\toval(\expr_1) = \bot \and
   \All \state_1. \red(\expr_1, \state_1) \\
   \All \state_1, \expr_2', \state'_2, \bar\expr'. \expr_1,\state_1 \step \expr'_2,\state'_2,\bar\expr' \Ra \state_1 = \state'_2 \land \expr_2 = \expr_2' \land \bar\expr = \bar\expr'}
  {\later \Bigl( \wpre{\expr_2}[\mask_1]{\Ret\var.\prop} * \Sep_{\expr_\f \in \bar\expr} \wpre{\expr_\f}[\top]{\Ret\any.\TRUE} \Bigr) \proves \wpre{\expr_1}[\mask_1]{\Ret\var.\prop}}
\end{mathparpagebreakable}


\paragraph{Adequacy of weakest precondition.}

The adequacy statement concerning functional correctness reads as follows:
\begin{align*}
 &\All \mask, \expr, \val, \pred, \state, \melt, \state', \tpool'.
 \\&(\All n. \melt \in \mval_n) \Ra
 \\&( \ownPhys\state * \ownM\melt \proves \wpre{\expr}[\mask]{x.\; \pred(x)}) \Ra
 \\&\cfg{\state}{[\expr]} \step^\ast
     \cfg{\state'}{[\val] \dplus \tpool'} \Ra
     \\&\pred(\val)
\end{align*}
where $\pred$ is a \emph{meta-level} predicate over values, \ie it can mention neither resources nor invariants.
\ralf{TODO: We can't just embed meta-level predicates like this. After all, this is a deep embedding.}

Furthermore, the following adequacy statement shows that our weakest preconditions imply that the execution never gets \emph{stuck}: Every expression in the thread pool either is a value, or can reduce further.
\begin{align*}
 &\All \mask, \expr, \state, \melt, \state', \tpool'.
 \\&(\All n. \melt \in \mval_n) \Ra
 \\&( \ownPhys\state * \ownM\melt \proves \wpre{\expr}[\mask]{x.\; \pred(x)}) \Ra
 \\&\cfg{\state}{[\expr]} \step^\ast
     \cfg{\state'}{\tpool'} \Ra
     \\&\All\expr'\in\tpool'. \toval(\expr') \neq \bot \lor \red(\expr', \state')
\end{align*}
Notice that this is stronger than saying that the thread pool can reduce; we actually assert that \emph{every} non-finished thread can take a step.

\paragraph{Hoare triples.}
It turns out that weakest precondition is actually quite convenient to work with, in particular when perfoming these proofs in Coq.
Still, for a more traditional presentation, we can easily derive the notion of a Hoare triple:
\[
\hoare{\prop}{\expr}{\Ret\val.\propB}[\mask] \eqdef \always{(\prop \Ra \wpre{\expr}[\mask]{\Ret\val.\propB})}
\]

We only give some of the proof rules for Hoare triples here, since we usually do all our reasoning directly with weakest preconditions and use Hoare triples only to write specifications.
\begin{mathparpagebreakable}
\inferH{Ht-ret}
  {}
  {\hoare{\TRUE}{\valB}{\Ret\val. \val = \valB}[\mask]}
\and
\inferH{Ht-bind}
  {\text{$\lctx$ is a context} \and \hoare{\prop}{\expr}{\Ret\val. \propB}[\mask] \\
   \All \val. \hoare{\propB}{\lctx(\val)}{\Ret\valB.\propC}[\mask]}
  {\hoare{\prop}{\lctx(\expr)}{\Ret\valB.\propC}[\mask]}
\and
\inferH{Ht-csq}
  {\prop \vs \prop' \\
    \hoare{\prop'}{\expr}{\Ret\val.\propB'}[\mask] \\   
   \All \val. \propB' \vs \propB}
  {\hoare{\prop}{\expr}{\Ret\val.\propB}[\mask]}
\and
% \inferH{Ht-mask-weaken}
%   {\hoare{\prop}{\expr}{\Ret\val. \propB}[\mask]}
%   {\hoare{\prop}{\expr}{\Ret\val. \propB}[\mask \uplus \mask']}
% \\\\
\inferH{Ht-frame}
  {\hoare{\prop}{\expr}{\Ret\val. \propB}[\mask]}
  {\hoare{\prop * \propC}{\expr}{\Ret\val. \propB * \propC}[\mask]}
\and
% \inferH{Ht-frame-step}
%   {\hoare{\prop}{\expr}{\Ret\val. \propB}[\mask] \and \toval(\expr) = \bot \and \mask_2 \subseteq \mask_2 \\\\ \propC_1 \vs[\mask_1][\mask_2] \later\propC_2 \and \propC_2 \vs[\mask_2][\mask_1] \propC_3}
%   {\hoare{\prop * \propC_1}{\expr}{\Ret\val. \propB * \propC_3}[\mask \uplus \mask_1]}
% \and
\inferH{Ht-atomic}
  {\prop \vs[\mask \uplus \mask'][\mask] \prop' \\
    \hoare{\prop'}{\expr}{\Ret\val.\propB'}[\mask] \\   
   \All\val. \propB' \vs[\mask][\mask \uplus \mask'] \propB \\
   \physatomic{\expr}
  }
  {\hoare{\prop}{\expr}{\Ret\val.\propB}[\mask \uplus \mask']}
\and
\inferH{Ht-false}
  {}
  {\hoare{\FALSE}{\expr}{\Ret \val. \prop}[\mask]}
\and
\inferHB{Ht-disj}
  {\hoare{\prop}{\expr}{\Ret\val.\propC}[\mask] \and \hoare{\propB}{\expr}{\Ret\val.\propC}[\mask]}
  {\hoare{\prop \lor \propB}{\expr}{\Ret\val.\propC}[\mask]}
\and
\inferHB{Ht-exist}
  {\All \var. \hoare{\prop}{\expr}{\Ret\val.\propB}[\mask]}
  {\hoare{\Exists \var. \prop}{\expr}{\Ret\val.\propB}[\mask]}
\and
\inferHB{Ht-box}
  {\always\propB \proves \hoare{\prop}{\expr}{\Ret\val.\propC}[\mask]}
  {\hoare{\prop \land \always{\propB}}{\expr}{\Ret\val.\propC}[\mask]}
% \and
% \inferH{Ht-inv}
%   {\hoare{\later\propC*\prop}{\expr}{\Ret\val.\later\propC*\propB}[\mask] \and
%    \physatomic{\expr}
%   }
%   {\knowInv\iname\propC \proves \hoare{\prop}{\expr}{\Ret\val.\propB}[\mask \uplus \set\iname]}
% \and
% \inferH{Ht-inv-timeless}
%   {\hoare{\propC*\prop}{\expr}{\Ret\val.\propC*\propB}[\mask] \and
%    \physatomic{\expr} \and \timeless\propC
%   }
%   {\knowInv\iname\propC \proves \hoare{\prop}{\expr}{\Ret\val.\propB}[\mask \uplus \set\iname]}
\end{mathparpagebreakable}

\subsection{Invariant Namespaces}
\label{sec:namespaces}

In \Sref{sec:invariants}, we defined an assertion $\knowInv\iname\prop$ expressing knowledge (\ie the assertion is persistent) that $\prop$ is maintained as invariant with name $\iname$.
The concrete name $\iname$ is picked when the invariant is allocated, so it cannot possibly be statically known -- it will always be a variable that's threaded through everything.
However, we hardly care about the actual, concrete name.
All we need to know is that this name is \emph{different} from the names of other invariants that we want to open at the same time.
Keeping track of the $n^2$ mutual inequalities that arise with $n$ invariants quickly gets in the way of the actual proof.

To solve this issue, instead of remembering the exact name picked for an invariant, we will keep track of the \emph{namespace} the invariant was allocated in.
Namesapces are sets of invariants, following a tree-like structure:
Think of the name of an invariant as a sequence of identifiers, much like a fully qualified Java class name.
A \emph{namespace} $\namesp$ then is like a Java package: it is a sequence of identifiers that we think of as \emph{containing} all invariant names that begin with this sequence. For example, \texttt{org.mpi-sws.iris} is a namespace containing the invariant name \texttt{org.mpi-sws.iris.heap}.

The crux is that all namespaces contain infinitely many invariants, and hence we can \emph{freely pick} the namespace an invariant is allocated in -- no further, unpredictable choice has to be made.
Furthermore, we will often know that namespaces are \emph{disjoint} just by looking at them.
The namespaces $\namesp.\texttt{iris}$ and $\namesp.\texttt{gps}$ are disjoint no matter the choice of $\namesp$.
As a result, there is often no need to track disjointness of namespaces, we just have to pick the namespaces that we allocate our invariants in accordingly.

Formally speaking, let $\namesp \in \textlog{InvNamesp} \eqdef \textlog{list}(\mathbb N)$ be the type of \emph{invariant namespaces}.
We use the notation $\namesp.\iname$ for the namespace $[\iname] \dplus \namesp$.
(In other words, the list is ``backwards''. This is because cons-ing to the list, like the dot does above, is easier to deal with in Coq than appending at the end.)

The elements of a namespaces are \emph{structured invariant names} (think: Java fully qualified class name).
They, too, are lists of $\mathbb N$, the same type as namespaces.
In order to connect this up to the definitions of \Sref{sec:invariants}, we need a way to map structued invariant names to $\mathbb N$, the type of ``plain'' invariant names.
Any injective mapping $\textlog{namesp\_inj}$ will do; and such a mapping has to exist because $\textlog{list}(\mathbb N)$ is countable.
Whenever needed, we (usually implicitly) coerce $\namesp$ to its encoded suffix-closure, \ie to the set of encoded structured invariant names contained in the namespace: \[\namecl\namesp \eqdef \setComp{\iname}{\Exists \namesp'. \iname = \textlog{namesp\_inj}(\namesp' \dplus \namesp)}\]

We will overload the notation for invariant assertions for using namespaces instead of names:
\[ \knowInv\namesp\prop \eqdef \Exists \iname \in \namecl\namesp. \knowInv\iname{\prop} \]
We can now derive the following rules (this involves unfolding the definition of view updates):
\begin{mathpar}
  \axiomH{inv-persist}{\knowInv\namesp\prop \proves \always\knowInv\namesp\prop}

  \axiomH{inv-alloc}{\later\prop \proves \pvs[\namesp] \knowInv\namesp\prop}

  \inferH{inv-open}
  {\namesp \subseteq \mask}
  {\knowInv\namesp\prop \vs[\mask][\mask\setminus\namesp] \later\prop * (\later\prop \vsW[\mask\setminus\namesp][\mask] \TRUE)}

  \inferH{inv-open-timeless}
  {\namesp \subseteq \mask \and \timeless\prop}
  {\knowInv\namesp\prop \vs[\mask][\mask\setminus\namesp] \prop * (\prop \vsW[\mask\setminus\namesp][\mask] \TRUE)}
\end{mathpar}


%%% Local Variables:
%%% mode: latex
%%% TeX-master: "iris"
%%% End:
