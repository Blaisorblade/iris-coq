%%%%%%%%%%%%%%%%%%%%%%%%%%%%%%%%%%%%%%%%%%%%%%%%%%%%%%%%%%%%%%%%
% PACKAGES
%%%%%%%%%%%%%%%%%%%%%%%%%%%%%%%%%%%%%%%%%%%%%%%%%%%%%%%%%%%%%%%%
\usepackage{mathtools}
%\usepackage{amsmath}
\usepackage{amsfonts}
\usepackage{amsthm}
\usepackage{amssymb}
\usepackage{stmaryrd}

\usepackage{mathpartir}

\usepackage{array}
\usepackage{tabu}

\usepackage{dashbox}

\usepackage{pftools}

\usepackage{xcolor}  % for print version

\usepackage{graphicx}
\usepackage{tikz}
\usepackage{scalerel}

\usepackage{rotating}
\usepackage{xparse}
\usepackage{xstring}
\usepackage{semantic}
\usepackage{csquotes}

\usepackage{hyperref}

%%%%%%%%%%%%%%%%%%%%%%%%%%%%%%%%%%%%%%%%%%%%%%%%%%%%%%%%%%%%%%%%
% SETUP
%%%%%%%%%%%%%%%%%%%%%%%%%%%%%%%%%%%%%%%%%%%%%%%%%%%%%%%%%%%%%%%%

\extrarowheight=\jot	% else, arrays are scrunched compared to, say, aligned
\newcolumntype{.}{@{}}
% Array {rMcMl} modifies array {rcl}, putting mathrel-style spacing
% around the centered column. (We used this, for example, in laying
% out some of Iris' axioms. Generally, aligned is simpler but aligned
% does not work in mathpar because \\ inherits mathpar's 2em vskip.)
% The capital M stands for THICKMuskip. The smaller medmuskip would be
% right for mathbin-style spacing.
\newcolumntype{M}{@{\mskip\thickmuskip}}

\definecolor{StringRed}{rgb}{.637,0.082,0.082}
\definecolor{CommentGreen}{rgb}{0.0,0.55,0.3}
\definecolor{KeywordBlue}{rgb}{0.0,0.3,0.55}
\definecolor{LinkColor}{rgb}{0.55,0.0,0.3}
\definecolor{CiteColor}{rgb}{0.55,0.0,0.3}
\definecolor{HighlightColor}{rgb}{0.0,0.0,0.0}

\usetikzlibrary{shapes}
%\usetikzlibrary{snakes}
\usetikzlibrary{arrows}
\usetikzlibrary{calc}
\usetikzlibrary{arrows.meta}
\tikzstyle{state}=[circle, draw, minimum size=1.2cm, align=center]
\tikzstyle{trans}=[arrows={->[scale=1.4]}]

\tikzstyle{layer}=[rounded corners=2pt, thin, align=center, draw, minimum width=4.2cm,minimum height=0.8cm]

\definecolor{grey}{rgb}{0.5,0.5,0.5}
\definecolor{red}{rgb}{1,0,0}

\hypersetup{%
  linktocpage=true, pdfstartview=FitV,
  breaklinks=true, pageanchor=true, pdfpagemode=UseOutlines,
  plainpages=false, bookmarksnumbered, bookmarksopen=true, bookmarksopenlevel=3,
  hypertexnames=true, pdfhighlight=/O,
  colorlinks=true,linkcolor=LinkColor,citecolor=CiteColor,
  urlcolor=LinkColor
}


%\theoremstyle{definition}
%\newtheorem{prop}{Prop}
\newtheorem{defn}{Definition}
\newtheorem{cor}{Corollary}
\newtheorem{conj}{Conj}
\newtheorem{lem}{Lemma}
\newtheorem{thm}{Theorem}

\newtheorem{exercise}{Exercise}
%%%%%%%%%%%%%%%%%%%%%%%%%%%%%%%%%%%%%%%%%%%%%%%%%%%%%%%%%%%%%%%%
% FONTS & FORMATTING
%%%%%%%%%%%%%%%%%%%%%%%%%%%%%%%%%%%%%%%%%%%%%%%%%%%%%%%%%%%%%%%%
\SetSymbolFont{stmry}{bold}{U}{stmry}{m}{n} % this fixes warnings when \boldsymbol is used with stmaryrd included

\newcommand{\textdom}[1]{\textit{#1}}
\newcommand{\textlog}[1]{\textsf{#1}}
\newcommand{\textsort}[1]{\textlog{#1}}
\newcommand{\textlang}[1]{\texttt{#1}}
\newcommand{\textmon}[1]{\textsc{#1}}


%%%%%%%%%%%%%%%%%%%%%%%%%%%%%%%%%%%%%%%%%%%%%%%%%%%%%%%%%%%%%%%%
% GENERIC MACROS
%%%%%%%%%%%%%%%%%%%%%%%%%%%%%%%%%%%%%%%%%%%%%%%%%%%%%%%%%%%%%%%%
\newcommand*{\Sref}[1]{\hyperref[#1]{\S\ref*{#1}}}
\newcommand*{\secref}[1]{\hyperref[#1]{Section~\ref*{#1}}}
\newcommand*{\lemref}[1]{\hyperref[#1]{Lemma~\ref*{#1}}}
\newcommand{\corref}[1]{\hyperref[#1]{Cor.~\ref*{#1}}}
\newcommand*{\defref}[1]{\hyperref[#1]{Definition~\ref*{#1}}}
\newcommand*{\egref}[1]{\hyperref[#1]{Example~\ref*{#1}}}
\newcommand*{\appendixref}[1]{\hyperref[#1]{Appendix~\ref*{#1}}}
\newcommand*{\figref}[1]{\hyperref[#1]{Figure~\ref*{#1}}}
\newcommand*{\tabref}[1]{\hyperref[#1]{Table~\ref*{#1}}}

\newcommand{\changes}{{\bf\color{red}{Changes}}}
\newcommand{\TODO}{\vskip 4pt {\color{red}\bf TODO}}

%\newcommand{\bigast}{\scalebox{3}{\raisebox{-0.3ex}{$\ast$}}}
%\newcommand{\bigtimes}{\scalebox{2.5}{\raisebox{-0.3ex}{$\times$}}}
\DeclareMathOperator*{\Sep}{\scalerel*{\ast}{\sum}}
\newcommand{\bigast}{\Sep}

\newcommand*{\sep}[1][]{\mathrel{\#_{#1}}}	% bad name; it's a different "sep"

\newcommand{\ALT}{\ |\ }

\newenvironment{pf}
  {\resetpfcounter\begin{proof}}
  {\end{proof}}

% superscript to the left
\def\presuper#1#2%
  {\mathop{}%
   \mathopen{\vphantom{#2}}^{#1}%
   \kern-\scriptspace%
   #2}


\newcommand{\upclose}{\mathord{\uparrow}}

%%%%%%%%%%%%%%%%%%%%%%%%%%%%%%%%%%%%%%%%%%%%%%%%%%%%%%%%%%%%%%%%
% LANGUAGE-LEVEL SYNTAX AND SEMANTICS
%%%%%%%%%%%%%%%%%%%%%%%%%%%%%%%%%%%%%%%%%%%%%%%%%%%%%%%%%%%%%%%%

\newcommand{\cfg}[2]{{#1};{#2}}
\newcommand{\fork}[1]{\textlang{fork}\;{#1}}

   
%%%%%%%%%%%%%%%%%%%%%%%%%%%%%%%%%%%%%%%%%%%%%%%%%%%%%%%%%%%%%%%%
% METAVARIABLES
%%%%%%%%%%%%%%%%%%%%%%%%%%%%%%%%%%%%%%%%%%%%%%%%%%%%%%%%%%%%%%%%
\newcommand{\aexpr}{a}
\newcommand{\expr}{e}
\newcommand{\type}{\tau}
\newcommand{\htype}{\sigma}
\newcommand{\ctype}{\sigma}
\newcommand{\heap}{h}
\newcommand{\tyvar}{\alpha}
\newcommand{\tyvarB}{\beta}
\newcommand{\val}{v}
\newcommand{\valB}{w}
\newcommand{\hval}{u}
\newcommand{\tls}{L}
\newcommand{\tlsVar}{L}

\newcommand{\cenv}{\Omega}
\newcommand{\tenv}{\Gamma}
\newcommand{\tvenv}{\Delta}

%\newcommand{\vctx}{\mathcal{X}}
\newcommand{\pvar}{p}
\newcommand{\pvarB}{q}
%\newcommand{\pvarC}{r}

\newcommand{\ectx}{K}
\newcommand{\tpool}{T}

% \newcommand{\progexpr}{p}
% \newcommand{\progctx}{D}

\newcommand{\subst}{\gamma}

%\newcommand{\island}{I}
\newcommand{\sisland}{\iota}
%\newcommand{\islands}{\omega}
%\newcommand{\islands}{\mathbf{\island}}

\newcommand{\predinterp}{\PRED}
\newcommand{\propinterp}{\mathcal{P}}

\newcommand{\PROP}{\mathcal{P}}
\newcommand{\PROPB}{\mathcal{Q}}

\newcommand{\interp}{\textrm{interp}}
\newcommand{\interps}{\textrm{interpAll}}

\newcommand{\restype}{\theta}
\newcommand{\restypes}{\boldsymbol{\theta}}


%\newcommand{\aprop}{{\color{red}A}}

\newcommand{\prop}{P}
\newcommand{\propB}{Q}
\newcommand{\propC}{R}

\newcommand{\pred}{\varphi}
\newcommand{\predB}{\psi}
\newcommand{\predC}{\zeta}

% \newcommand{\Prop}{\mathbb{B}}
% \newcommand{\Pred}{\mathbb{P}}

\newcommand{\rs}{r}
\newcommand{\rsB}{s}

%\newcommand{\propSet}{\mathcal{P}}
%\newcommand{\apropSet}{\mathcal{A}}
%\newcommand{\pfctx}{\mathcal{C}}
\newcommand{\vctx}{\Gamma}
\newcommand{\pfctx}{\Theta}


\newcommand{\assert}{\varphi}
\newcommand{\assertB}{\psi}

\newcommand{\PRED}{\Phi}

%% \newcommand{\pname}{\pi}
%% \newcommand{\prot}{\pi}
%% \newcommand{\prots}{\boldsymbol{\pi}}
%% \newcommand{\protSet}{\mathcal{N}}

\newcommand{\iname}{\iota}
\newcommand{\inameB}{\iota'}
\newcommand{\inv}{I}
\newcommand{\invs}{\mathcal{I}}
\newcommand{\mask}{\mathcal{E}}

\newcommand{\state}{\varsigma}
\newcommand{\prescar}{\Pi}
\newcommand{\pres}{\pi}

\newcommand{\var}{x}
\newcommand{\varB}{y}
\newcommand{\varC}{z}
%\newcommand{\VAL}{d}
\newcommand{\ectxVar}{\kappa}

\newcommand{\term}{t}
\newcommand{\termB}{u}
\newcommand{\termVal}{V}

\newcommand{\sort}{\sigma}

\newcommand{\SigNat}{\Sigma}
\newcommand{\SigType}{\mathcal{T}}
\newcommand{\SigFn}{\mathcal{F}}
\newcommand{\sigfn}{F}

\newcommand{\tmap}{B}
\newcommand{\ttokSet}{I}

\newcommand{\monoid}{M}

%\newcommand{\mvar}{a}
%\newcommand{\mvarB}{b}
\newcommand{\melt}{a}
\newcommand{\meltB}{b}
\newcommand{\meltC}{c}
\newcommand{\melts}{A}
\newcommand{\meltsB}{B}
\newcommand{\ghostRes}{g}

\newcommand{\gname}{\gamma}

%%%%%%%%%%%%%%%%%%%%%%%%%%%%%%%%%%%%%%%%%%%%%%%%%%%%%%%%%%%%%%%%
% IDENTIFIERS
%%%%%%%%%%%%%%%%%%%%%%%%%%%%%%%%%%%%%%%%%%%%%%%%%%%%%%%%%%%%%%%%


\newcommand{\Prop}{\textlog{Prop}}
\newcommand{\Pred}{\textlog{Pred}}

\newcommand{\PropDom}{\textdom{Prop}}
\newcommand{\PredDom}{\textdom{Pred}}


%%%%%%%%%%%%%%%%%%%%%%%%%%%%%%%%%%%%%%%%%%%%%%%%%%%%%%%%%%%%%%%%
% SYMBOLS, NOTATION
%%%%%%%%%%%%%%%%%%%%%%%%%%%%%%%%%%%%%%%%%%%%%%%%%%%%%%%%%%%%%%%%

\def\All #1.{\forall #1.\;}%
\def\Exists #1.{\exists #1.\;}%
\def\Absp #1.{({#1}).\;}
\def\Ret #1.{#1.\;}%
\def\Lam #1.{\lambda #1.\;}%
\def\MU #1.{\mu #1.\;}%
\newcommand{\any}{{\rule[-.2ex]{1ex}{.4pt}}}%
\newcommand{\unitval}{()}%


\newcommand{\fullSat}[4]{#1 \models_{#2} #3; #4}
\newcommand{\fullNSat}[6]{#2 \models_{#3}^{#1} #4; #5; #6}

\newcommand{\erasestate}[1]{|#1|_\state}
\newcommand{\eraseexp}[1]{|#1|_\expr}

\newcommand{\mcar}[1]{|#1|}
\newcommand{\mcarp}[1]{\mcar{#1}^{+}}
\newcommand{\mzero}{\bot}
\newcommand{\munit}{\mathord{\varepsilon}}
\newcommand{\mtimes}{\mathbin{\cdot}}

\newcommand{\gtimes}{\bullet}



\newcommand{\protAt}[3]{\mbox{$
		\begin{array}{|@{}l@{}|@{}l@{}|}
		\firsthline \;#1 : #2\;\; & \;#3\;\; \\
		\lasthline
		\end{array}$}}
\newcommand{\protAtB}[2]{\mbox{$
		\begin{array}{|@{}l@{}|}
		\firsthline \;#1 : #2\;\; \\
		\lasthline
		\end{array}$}}

% PDS: The baseline of the boxed contents of
% \oldKnowInv and \oldOwnGGhost and \oldOwnGhost isn't right:
% It can be lower than the surrounding formula.
%\newcommand{\oldKnowInv}[2]{\mbox{$
%  \begin{array}{|@{\;}c@{\;}|}
%     \firsthline #2 \\
%     \lasthline
%  \end{array}$}{}^{\,#1}}

%% \newcommand{\ownGhost}[2]{{\dbox{$#1 : #2$}}}
%% \newcommand{\ownGhostB}[3]{\dbox{$#1 : #2$}{}_{#3}}

% \newcommand{\ownGhost}[3]{\mbox{$
%   \begin{array}{:@{}l@{}:@{}l@{}:}
%      \firsthdashline \;#1 : #2\;\; & \;#3\;\; \\
%      \lasthdashline
%   \end{array}$}}
%\newcommand{\oldOwnGhost}[2]{\mbox{$  
%  \begin{array}{:@{\;}c@{\;}:}
%     \firsthdashline #2 \\
%     \lasthdashline
%  \end{array}$}{}^{\,#1}}
%\newcommand{\oldOwnGGhost}[1]{\mbox{$  
%  \begin{array}{:@{\;}c@{\;}:}
%     \firsthdashline #1 \\
%     \lasthdashline
%  \end{array}$}}

% PDS: Was 0pt inner, 2pt outer.
% \boxedassert [tikzoptions] contents [name]
\tikzstyle{boxedassert_border} = [sharp corners,line width=0.2pt]
\NewDocumentCommand \boxedassert {O{} m o}{%
	\tikz[baseline=(m.base)]{
		%	  \node[rectangle, draw,inner sep=0.8pt,anchor=base,#1] (m) {${#2}\mathstrut$};
		\node[rectangle,inner sep=0.8pt,outer sep=0.2pt,anchor=base] (m) {${#2}\mathstrut$};
		\draw[#1,boxedassert_border] ($(m.south west) + (0,0.65pt)$) rectangle ($(m.north east) + (0, 0.7pt)$);
	}\IfNoValueF{#3}{^{\,#3}}%
}
\newcommand*{\knowInv}[2]{\boxedassert{#2}[#1]}
\newcommand*{\ownGhost}[2]{\boxedassert[densely dashed]{#2}[#1]}
\newcommand*{\ownGGhost}[1]{\boxedassert[densely dashed]{#1}}

\newcommand{\ownPhys}[1]{\lfloor#1\rfloor}

\newcommand{\supported}[1]{\left[ #1 \right]}


%\newcommand*{\know}[2]{\knowInv{#1}{#2}}%
%\newcommand*{\own}[2]{\ownGhost{#1}{#2}}%

%\newcommand{\varset}{\mathcal{X}}

\newcommand{\simpl}{\textsc{i}}
\newcommand{\sspec}{\textsc{s}}
\newcommand{\IMSP}{\simpl\sspec}

\newcommand{\pointsto}{\hookrightarrow}
\newcommand{\wand}{\;{{\mbox{---}}\!\!{*}}\;}
%\newcommand{\gm}{\Rrightarrow}

\NewDocumentCommand \vsGen {O{} m O{}}%
  {\mathrel{%
    \ifthenelse{\equal{#3}{}}{%
      % Just one mask, or none
      {#2}_{#1}%
    }{%
      % Two masks
      \presuper{#1}{#2}^{#3}
    }%
  }}%
\NewDocumentCommand \vs {O{} O{}} {\vsGen[#1]{\Rrightarrow}[#2]}
\NewDocumentCommand \vsL {O{} O{}} {\vsGen[#1]{\Lleftarrow}[#2]}
\NewDocumentCommand \vsE {O{} O{}} %
  {\vsGen[#1]{\Lleftarrow\!\!\!\Rrightarrow}[#2]}

\newcommand{\mupd}{\rightsquigarrow}

\newcommand{\heapmaps}[1]{\hookrightarrow_{#1}}
\newcommand{\codemaps}[1]{\Mapsto_{#1}}

\newcommand{\implmaps}{\heapmaps{\IM}}
\newcommand{\implmapscode}{\codemaps{\IM}}

\newcommand{\specmaps}{\heapmaps{\SP}}
\newcommand{\specmapscode}{\codemaps{\SP}}

\newcommand{\IM}{\simpl}
\newcommand{\SP}{\sspec}

%\newcommand{\tRole}[1]{\texttok{Tid}(#1)}
\newcommand{\bij}[2]{{#1} \bowtie {#2}}

\newcommand{\iassert}[3]{\fbox{$#1$}{}^{#2}_{#3}}

%\newcommand{\mown}[2]{\textsf{own}(#1, #2)}
\newcommand{\mown}[3]{\fbox{$#1$}^{#2}_{#3}}
\newcommand{\minterp}[2]{\textsf{interp}(#1) = #2}
\newcommand{\mdisable}[2]{\delta_{#1}(#2)}

\newcommand{\TRUE}{\textlog{True}}
\newcommand{\FALSE}{\textlog{False}}
\newcommand{\emp}{\textsf{emp}}

\newcommand{\const}{\textlog{Inv}}

\newcommand{\infinite}{\textlog{infinite}}

\newcommand{\tokPure}{\textlog{TokPure}}
\newcommand{\timeless}[1]{\textlog{timeless}(#1)}

\newcommand{\physatomic}[1]{\text{$#1$ phys.\ atomic}}

\newcommand{\unlimRely}[1]{\cdot \geqRely {#1}}

\newcommand{\fmapsto}[1][-]{\stackrel{#1}{\mapsto}}
\newcommand{\gmapsto}{\hookrightarrow}%
\newcommand{\fgmapsto}[1][-]{\stackrel{#1}{\gmapsto}}%


%%%%%%%%%%%%%%%%%%%%%%%%%%%%%%%%%%%%%%%%%%%%%%%%%%%%%%%%%%%%%%%%
% JUDGMENTS
%%%%%%%%%%%%%%%%%%%%%%%%%%%%%%%%%%%%%%%%%%%%%%%%%%%%%%%%%%%%%%%%

\newcommand{\wfte}[1]{{#1}\;\mbox{tyenv}}
\newcommand{\wtt}[2]{#1 : #2}
\newcommand{\wt}[3]{#1 \proves #2 : #3}
\newcommand{\wtd}[4]{#1; #2 \proves #3 : #4}

\newcommand{\judgment}[2]{\paragraph{#1}\hspace{\stretch{1}}\fbox{$#2$}}
\newcommand{\judgmentB}[2]{\paragraph{#1}\hspace{\stretch{1}}{$#2$}}
\newcommand{\judgmentC}[2]{{\normalsize\textbf{\emph{#1}}}\hspace{\stretch{1}}{\fbox{$#2$}}}
\newcommand{\judgmentD}[2]{{\normalsize\textbf{\emph{#1}}}\quad{\fbox{$#2$}}}

\newcommand{\isAtomic}[2]{\cfg{#1}{#2}\ \textrm{atomic}}

%%%%%%%%%%%%%%%%%%%%%%%%%%%%%%%%%%%%%%%%%%%%%%%%%%%%%%%%%%%%%%%%
% MATH SYMBOLS & NOTATION
%%%%%%%%%%%%%%%%%%%%%%%%%%%%%%%%%%%%%%%%%%%%%%%%%%%%%%%%%%%%%%%%

\newcommand{\pfn}{\rightharpoonup}
\newcommand{\fpfn}{\stackrel{\textrm{fin}}{\rightharpoonup}}
\newcommand{\ra}{\rightarrow}
\newcommand{\Ra}{\Rightarrow}
\newcommand{\Lra}{\Leftrightarrow}
\newcommand{\monra}{\stackrel{\textrm{mon}}{\rightarrow}}
%\newcommand{\res}{\upharpoonright}
\newcommand{\step}{\ra}
%\newcommand{\monora}{\stackrel{\textrm{mono}}{\longrightarrow}}
\newcommand{\monora}{\Rightarrow}

\newcommand{\restr}[2]{\lfloor #1 \rfloor_{#2}}
% LB
\newcommand{\nequiv}[1]{\ensuremath{\mathrel{\stackrel{#1}{=}}}}
\newcommand{\notnequiv}[1]{\ensuremath{\mathrel{\stackrel{#1}{\neq}}}}
\newcommand{\nequivset}[2]{\ensuremath{\mathrel{\stackrel{#1}{=}_{#2}}}}
\newcommand{\nequivB}[1]{\ensuremath{\mathrel{\stackrel{#1}{\equiv}}}}
\newcommand{\UPred}{\textdom{UPred}}
\newcommand{\SPred}{\textdom{SPred}}
\newcommand{\latert}{\mathord{\blacktriangleright}}

%\newcommand{\emp}{1}
% \newcommand{\lget}{\textrm{get}}
% \newcommand{\lput}{\textrm{put}}
% \newcommand{\trans}{\textrm{trans}}

\newcommand{\lift}[1]{\lfloor {#1} \rfloor}

\newcommand{\Sem}[1]{\llbracket #1 \rrbracket}

\newcommand{\semSort}[1]{\Sem{#1}}
\newcommand{\semTerm}[1]{\Sem{#1}}
\newcommand{\semVCtx}[1]{\Sem{#1}}
\newcommand{\semProtSet}[1]{\Sem{#1}}
\newcommand{\semAProp}[1]{\Sem{#1}}

%\newcommand{\semProp}[3]{#1 \models^{#2} #3}
\newcommand{\semProp}[3]{#1 \in \llbracket #3 \rrbracket^{#2}}
\newcommand{\semPropB}[2]{\llbracket #2 \rrbracket^{#1}}
\newcommand{\semPred}[2]{\llbracket{#1}\rrbracket^{#2}}

\newcommand{\Interp}[1]{\mathcal{I}\llbracket #1 \rrbracket}
\newcommand{\Val}[1]{\llbracket #1 \rrbracket}
\newcommand{\ValB}{\mathbb{V}}
\newcommand{\LiftVal}[1]{\widehat{\mathcal{V}}\llbracket #1 \rrbracket}
\newcommand{\Exp}[1]{\mathcal{E}\llbracket #1 \rrbracket}
\newcommand{\LiftExp}[1]{\widehat{\mathcal{E}}\llbracket #1 \rrbracket}

\newcommand{\expPred}[3]{(#1, #2) \downarrow #3}
\newcommand{\expPredPure}[3]{(#1, #2) \downarrow^{\textrm{pure}} #3}

\newcommand{\Store}[1]{\mathcal{H}\llbracket #1 \rrbracket}
\newcommand{\Heap}[1]{\mathcal{H}\llbracket #1 \rrbracket}
\newcommand{\Env}[1]{\mathcal{G}\llbracket #1 \rrbracket}
\newcommand{\TEnv}[1]{\mathcal{D}\llbracket #1 \rrbracket}
\newcommand{\Ctx}[1]{\mathcal{K}\llbracket #1 \rrbracket}
% \newcommand{\Thread}[1]{\mathcal{T}\llbracket #1 \rrbracket}
% \newcommand{\ThreadRel}[3]{\mathcal{T}(#1,#2,#3)}
% \newcommand{\TRel}[4]{\mathcal{T}(#1,#2,#3,#4)}
% \newcommand{\TRelD}[1]{\mathcal{T}\llbracket{#1}\rrbracket}
\newcommand{\HVal}[1]{\mathcal{H}\llbracket #1 \rrbracket}
\newcommand{\Obs}[1]{\mathcal{O}(#1)}

\newcommand{\dyn}[2]{\textlog{wp}({#1}, {#2})}
\newcommand{\adyn}[2]{{#1}\;\llparenthesis{#2}\rrparenthesis}
\newcommand{\dynpred}[2]{\textdom{wp}({#1}, {#2})}
\newcommand{\dynA}[3]{\textlog{wp}_{#3}({#1}, {#2})}
\newcommand{\pvs}[1]{\textlog{vs}({#1})}
\newcommand{\pvsA}[3]{\textlog{vs}_{#2}^{#3}({#1})}


% \hoaresizebox pre post
% \hoarescalebox char sizebox
\newcommand*{\hoaresizebox}[1]{%
  \hbox{$\mathsurround=0pt{#1}\mathstrut$}}
\newcommand*{\hoarescalebox}[2]{%
  \hbox{\scalerel*[1ex]{#1}{#2}}}
\newcommand{\triple}[5]{%
  \setbox0=\hoaresizebox{{#3}{#5}}%
  \setbox1=\hoarescalebox{#1}{\copy0}%
  \setbox2=\hoarescalebox{#2}{\copy0}%
  \copy1{#3}\copy2%
  \;{#4}\;%
  \copy1{#5}\copy2}
\NewDocumentCommand \hoare {m m m O{}}{
	\triple\{\}{#1}{#2}{#3}%
	_{#4}%
}

\newcommand{\bracket}[4][]{%
  \setbox0=\hbox{$\mathsurround=0pt{#1}{#4}\mathstrut$}%
  \scalerel*[1ex]{#2}{\copy0}%
  {#4}%
  \scalerel*[1ex]{#3}{\copy0}}
% \curlybracket[other] x
\newcommand{\curlybracket}[2][]{\bracket[{#1}]\{\}{#2}}
\newcommand{\anglebracket}[2][]{\bracket[{#1}]\langle\rangle{#2}}
% \hoareV[t] pre c post [mask]
\NewDocumentCommand \hoareV {O{c} m m m O{}}{
		{\begin{aligned}[#1]
		&\curlybracket{#2} \\
		&\quad{#3} \\
		&{\curlybracket{#4}}_{#5}
		\end{aligned}}%
}
% \hoareHV[t] pre c post [mask]
\NewDocumentCommand \hoareHV {O{c} m m m O{}}{
	{\begin{aligned}[#1]
	&\curlybracket{#2} \; {#3} \\
	&{\curlybracket{#4}}_{#5}
	\end{aligned}}%
}

\newcommand{\ttrip}[4]{
  \semPropB{#1}{\rho}{\safe(#2, #3, #4)}
}
%% \newcommand{\ttrip}[4]{
%%   #1 \models^\rho 
%%     {#2}@{#3}\; 
%%   \big\{ #4 \big\}
%% }
\newcommand{\halfttrip}[3]{
    {#1}@{#2}\; 
  \big\{ #3 \big\}
}

\newcommand{\rewriteSpec}{\ra_\SP}

%\newcommand{\dyn}[2]{{#1}\;\{{#2}\}}

\newcommand{\safe}{\textsf{safe}}

\newcommand{\mthread}{m}
\newcommand{\absent}{\textsf{none}}

\newcommand{\PROG}[1]{\textrm{prog}\llbracket #1 \rrbracket}
\newcommand{\PRES}[1]{\textrm{pres}\llbracket #1 \rrbracket}

\newcommand{\pset}[1]{\wp(#1)}
\newcommand{\pmset}[1]{\wp_{m}(#1)}
\newcommand{\psetup}[1]{\wp^\uparrow(#1)}
\newcommand{\psetdown}[1]{\wp^\downarrow(#1)}
\newcommand{\fpset}[1]{\wp_{\textrm{fin}}(#1)}
\newcommand{\mset}[1]{\mathrm{bag}(#1)}
%\newcommand{\bag}[1]{\Lbag #1 \Rbag}
\newcommand{\eqdef}{\triangleq}

\newcommand{\extendseq}{\sqsupseteq}
\newcommand{\extends}{\sqsupset}
\newcommand{\beforeeq}{\sqsubseteq}
\newcommand{\extby}{\sqsubseteq}
%\newcommand{\ntime}{\triangleright}
\newcommand{\later}{\mathord{\triangleright}}
%\newcommand{\always}[1]{\Box{#1}}
\newcommand{\always}{\Box{}}
\newcommand{\dup}[1]{\textrm{dup}({#1})}
\newcommand{\restrict}[2]{\lfloor #1 \rfloor_{#2}}

\newcommand{\extendseqCtx}{\stackrel{\textrm{ctx}}{\sqsupseteq}}
\newcommand{\extbyCtx}{\stackrel{\textrm{ctx}}{\sqsubseteq}}

\newcommand{\leqWT}{\sqsubseteq}
\newcommand{\geqWT}{\sqsubseteq}
\newcommand{\lubWT}{\sqcup}

\newcommand{\leqRes}{\leq}
\newcommand{\geqRes}{\geq}

\newcommand{\geqIS}{\sqsupseteq}

\newcommand{\proves}{\vdash}
\newcommand{\provesalways}{\vdash_{\!\!\boxempty}}
\newcommand{\refines}{\leq}
\newcommand{\pbrk}{\mbox{\phantom{.}}}

\newcommand{\hasHVal}[3]{#1 \Ra \textrm{hval}(#2, #3)}


\newcommand{\dom}{\textrm{dom}}
\newcommand{\rng}{\textrm{rng}}
\newcommand{\cod}{\textrm{cod}}


\newcommand{\IF}{\mathrel{\text{if}}}
\newcommand{\WHEN}{\textrm{when }}

\newcommand{\ie}{\emph{i.e.,} }
\newcommand{\eg}{\emph{e.g.,} }
\newcommand{\etal}{\emph{et~al.}}
\newcommand{\wrt}{w.r.t.~}
\newcommand{\deadfootnote}[1]{}

\newcommand{\idisl}[2]{{#1} \mapsto {#2}}

%\newcommand{\region}[4]

\newcommand{\SET}[2]{
\left\{%
#1%
\;\middle|\;%
#2%
\right\}
}
\newcommand{\SETB}[1]{
\left\{%
#1%
\right\}
}
\newcommand{\SETC}[2]{#1 & #2}

\newcommand{\SPACER}{\;\;\;}

\newcommand{\wIso}{\xi}

\newcommand{\sembox}[1]{\hfill \normalfont \mbox{\fbox{\(#1\)}}}
\newcommand{\typedsection}[2]{\subsubsection*{\rm\em #1 \sembox{#2}}}

% what are we calling the manuscript?
\newcommand{\book}{book}

\newcommand{\aaron}[1]{{\color{red}\textbf{AT: #1}}}
\newcommand{\derek}[1]{{\color{red}\textbf{DD: #1}}}
\newcommand{\lars}[1]{{\color{red}\textbf{LB: #1}}}
\newcommand{\kasper}[1]{{\color{red}\textbf{KS: #1}}}
\newcommand{\ralf}[1]{{\color{red}\textbf{RJ: #1}}}
\newcommand{\dave}[1]{{\color{red}\textbf{PDS: #1}}}
\newcommand{\hush}[1]{}
\newcommand{\relaxguys}{%
	\let\aaron\hush%
	\let\derek\hush%
	\let\lars\hush%
	\let\kasper\hush%
	\let\ralf\hush%
	\let\dave\hush%
}

%%%%%%%%%%%%%%%%%%%%%%%%%%%%%%%%%%%%%%%%%%%%%%%%%%%%%%%%%%%%%%%%
% ATOMIC SHIFTS

\newcommand{\funnyforall}{\boldsymbol\forall}
\newcommand{\funnyexists}{\boldsymbol\exists}

\newcommand{\aspre}{P}
\newcommand{\asfrom}{\alpha}
\newcommand{\asto}{\beta}
\newcommand{\aspost}{Q}
%\newcommand{\asprop}{\aspost}
%\newcommand{\aspred}{\aspost}
\newcommand{\asframe}{R}
\newcommand{\asfmask}{\mask_{\asframe}}
\newcommand\nomask{\,\!}% to avoid \as defaults. It ain't empty, but it looks empty.

% we need 10 arguments, so use some magic to get that...
\newcommand\ascore[1]{%
    \def\tempflags{#1}%
    \ascoreContinued%
}
\newcommand{\ascoreContinued}[9]{
  {\stretchleftright[450]{\langle}{ %
  \IfSubStr{\tempflags}{l}{ \begin{inbox} }{} %
  #2
  \IfSubStr{\tempflags}{b}{\vsE}{\vs} %
  \IfSubStr{\tempflags}{a}{ #1.\;}{} %
  #3 %
  \IfSubStr{\tempflags}{x}{ \mid #4}{} %
  \IfSubStr{\tempflags}{f}{ %
    \mid %
    \IfSubStr{\tempflags}{l}{  \\ }{} %
    \IfSubStr{\tempflags}{e}{ #5.\;}{} %
    #6 \vs #7 %
  }{} %
  \IfSubStr{\tempflags}{l}{ \end{inbox} }{} %
  }{\rangle}}_{#9}^{#8} %
}
\NewDocumentCommand \as {d() m m o d() m m O{\top} O{}}
 { \ascore{ %
     \IfNoValueF{#1}{a} % universal quantifier
     b                  % arrow back to start
     \IfNoValueF{#4}{x} % explicit R and E
     f                  % add forwards shift to final state
     \IfNoValueF{#5}{e} % existential quantifier
   }{#1}{#2}{#3}{#4}{#5}{#6}{#7}{#8}{#9} %
 }
\NewDocumentCommand \asl {d() m m o d() m m O{\top} O{}}
 { \ascore{ %
     l                  % use multiple lines
     \IfNoValueF{#1}{a} % universal quantifier
     b                  % arrow back to start
     \IfNoValueF{#4}{x} % explicit R and E
     f                  % add forwards shift to final state
     \IfNoValueF{#5}{e} % existential quantifier
   }{#1}{#2}{#3}{#4}{#5}{#6}{#7}{#8}{#9} %
 }
% \NewDocumentCommand \am {d() m m o m}
%  { \ascore{ %
%      \IfNoValueF{#1}{a} % universal quantifier
%      b                  % arrow back to start
%      \IfNoValueF{#4}{x} % explicit R and E
%    }{#1}{#2}{#3}{#4}{#5}{}{}{} %
%  }

%%% Atomic triples

\NewDocumentCommand \ahoare {m m m O{} O{}}{
	\triple\langle\rangle{#1}{#2}{#3}%
	_{#5}^{#4}%
}

\NewDocumentCommand \ahoareV {O{c} m m m O{} O{}}{
		{\begin{aligned}[#1]
		&\anglebracket{#2} \\
		&\quad{#3} \\
		&{\anglebracket{#4}}_{#6}^{#5}
		\end{aligned}}%
}

\NewDocumentCommand \ahoareHV {O{c} m m m O{} O{}}{
	{\begin{aligned}[#1]
	&\anglebracket{#2}\;\;\; {#3} \\
	&{\anglebracket{#4}}_{#6}^{#5}
	\end{aligned}}%
}

%%%%%%%%%%%%%%%%%%%%%%%%%%%%%%%%%%%%%%%%%%%%%%%%%%%%%%%%%%%%%%%%
% hoare proof typesetting

\newenvironment{inbox}[1][]{
  \begin{array}[#1]{@{}l@{}}
}{
  \end{array}
}

\newcommand{\tabubox}[2][]{%
  \begin{tabu}{@{#1}X[1,l,m]@{}}%
    #2 %
  \end{tabu}%
}
\newcommand{\hproofnospace}[1]{\noindent\parbox{\linewidth}{#1}} %
\newcommand{\hproof}[1]{\vspace{0.5em}\hproofnospace{#1}\vspace{0.5em}} %
\newcommand\psub[2]{%
  \begin{tabu}{ m{0.9em} | X[1,l,m] }%
    \begin{sideways}#1\end{sideways} &%
    \tabubox{#2}%
  \end{tabu}%
}%

\newcommand\pind[1]{\tabubox[\hspace{1em}]{#1}}
\newcommand{\pline}[2][\empty]{\ensuremath{\left\{{#2\mathstrut}\right\}_{#1}}}
\newcommand{\pmline}[2][\empty]{\ensuremath{\left\{\begin{inbox}#2\end{inbox}\right\}_{#1}}}
\newcommand{\aline}[2][\empty]{\ensuremath{{\stretchleftright[450]{\langle}{#2\mathstrut}{\rangle}}_{#1}}}
\newcommand{\amline}[2][\empty]{\ensuremath{{\stretchleftright[450]{\langle}{\begin{inbox}#2\end{inbox}}{\rangle}}_{#1}}}
\definecolor{code_color}{rgb}{0, 0, 0.6}
\newcommand{\cdline}[1]{\ensuremath{\color{code_color}#1}}

\definecolor{interp_p_backgr}{rgb}{0.8, 0.8, 1.0}
\definecolor{interp_q_backgr}{rgb}{0.8, 1.0, 0.8}

%%%%%%%%%%%%%%%%%%%%%%%%%%%%%%%%%%%%%%%%%%%%%%%%%%%%%%%%%%%%%%%%
% Monoid and other constructions

\newcommand{\FHeap}{\textsc{FHeap}}
\newcommand{\AFHeap}{\textsc{AFHeap}}

\newcommand{\auth}[1]{\ensuremath{\textsc{Auth}(#1)}}
\newcommand{\authfull}{\mathord{\bullet}\,}
\newcommand{\authfrag}{\mathord{\circ}\,}

\newcommand{\fpfunm}[2]{\ensuremath{\textsc{FpFun}(#1, #2)}}
\newcommand{\fracm}[1]{\ensuremath{\textsc{Frac}(#1)}}
\newcommand{\exm}[1]{\ensuremath{\textsc{Ex}(#1)}}
\newcommand{\agm}[1]{\ensuremath{\textsc{Ag}(#1)}}

%\newcommand{\dispm}[1]{\ensuremath{\textsc{Disp}(#1)}}
%\newcommand{\disposed}{\mathord{\dagger}}


\newcommand{\STSMon}[1]{\textsc{Sts}_{#1}}
\newcommand{\STSInv}{\textsf{STSInv}}
\newcommand{\STS}{\textsf{STS}}
\newcommand{\STSS}{\mathcal{S}} % states
\newcommand{\STST}{\mathcal{T}} % tokens
\newcommand{\STSL}{\mathcal{L}} % labels
\newcommand{\ststrans}{\ra^{*}}%	the relation relevant to the STS rules

\newcommand{\AuthInv}{\textsf{AuthInv}}
\newcommand{\Auth}{\textsf{Auth}}
