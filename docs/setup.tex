
\makeatletter%
\@ifundefined{basedir}{%
\newcommand\basedir{}%
}{}%
\makeatother%
%%%%%%%%%%%%%%%%%%%%%%%%%%%%%%%%%%%%%%%%%%%%%%%%%%%%%%%%%%%%%%%%
%% PACKAGES
%%%%%%%%%%%%%%%%%%%%%%%%%%%%%%%%%%%%%%%%%%%%%%%%%%%%%%%%%%%%%%%%

%\usepackage{amsmath}
\usepackage{amsfonts}
\usepackage{amsthm}
\usepackage{amssymb}
\usepackage{stmaryrd}

\usepackage{mathpartir}

\usepackage{\basedir pftools}
\usepackage{\basedir iris}

\usepackage{xcolor}  % for print version

\usepackage{graphicx}
\usepackage{enumitem}
\usepackage{semantic}
\usepackage{csquotes}

\usepackage{hyperref}

%%%%%%%%%%%%%%%%%%%%%%%%%%%%%%%%%%%%%%%%%%%%%%%%%%%%%%%%%%%%%%%%
%% SETUP
%%%%%%%%%%%%%%%%%%%%%%%%%%%%%%%%%%%%%%%%%%%%%%%%%%%%%%%%%%%%%%%%
\SetSymbolFont{stmry}{bold}{U}{stmry}{m}{n} % this fixes warnings when \boldsymbol is used with stmaryrd included

\extrarowheight=\jot	% else, arrays are scrunched compared to, say, aligned
\newcolumntype{.}{@{}}
% Array {rMcMl} modifies array {rcl}, putting mathrel-style spacing
% around the centered column. (We used this, for example, in laying
% out some of Iris' axioms. Generally, aligned is simpler but aligned
% does not work in mathpar because \\ inherits mathpar's 2em vskip.)
% The capital M stands for THICKMuskip. The smaller medmuskip would be
% right for mathbin-style spacing.
\newcolumntype{M}{@{\mskip\thickmuskip}}

\definecolor{StringRed}{rgb}{.637,0.082,0.082}
\definecolor{CommentGreen}{rgb}{0.0,0.55,0.3}
\definecolor{KeywordBlue}{rgb}{0.0,0.3,0.55}
\definecolor{LinkColor}{rgb}{0.55,0.0,0.3}
\definecolor{CiteColor}{rgb}{0.55,0.0,0.3}
\definecolor{HighlightColor}{rgb}{0.0,0.0,0.0}

\definecolor{grey}{rgb}{0.5,0.5,0.5}
\definecolor{red}{rgb}{1,0,0}

\hypersetup{%
  linktocpage=true, pdfstartview=FitV,
  breaklinks=true, pageanchor=true, pdfpagemode=UseOutlines,
  plainpages=false, bookmarksnumbered, bookmarksopen=true, bookmarksopenlevel=3,
  hypertexnames=true, pdfhighlight=/O,
  colorlinks=true,linkcolor=LinkColor,citecolor=CiteColor,
  urlcolor=LinkColor
}


%\theoremstyle{definition}
%\newtheorem{prop}{Prop}
\newtheorem{defn}{Definition}
\newtheorem{cor}{Corollary}
\newtheorem{conj}{Conj}
\newtheorem{lem}{Lemma}
\newtheorem{thm}{Theorem}

\newtheorem{exercise}{Exercise}

%%%%%%%%%%%%%%%%%%%%%%%%%%%%%%%%%%%%%%%%%%%%%%%%%%%%%%%%%%%%%%%%
%% GENERIC MACROS
%%%%%%%%%%%%%%%%%%%%%%%%%%%%%%%%%%%%%%%%%%%%%%%%%%%%%%%%%%%%%%%%
\newcommand*{\Sref}[1]{\hyperref[#1]{\S\ref*{#1}}}
\newcommand*{\secref}[1]{\hyperref[#1]{Section~\ref*{#1}}}
\newcommand*{\lemref}[1]{\hyperref[#1]{Lemma~\ref*{#1}}}
\newcommand{\corref}[1]{\hyperref[#1]{Cor.~\ref*{#1}}}
\newcommand*{\defref}[1]{\hyperref[#1]{Definition~\ref*{#1}}}
\newcommand*{\egref}[1]{\hyperref[#1]{Example~\ref*{#1}}}
\newcommand*{\appendixref}[1]{\hyperref[#1]{Appendix~\ref*{#1}}}
\newcommand*{\figref}[1]{\hyperref[#1]{Figure~\ref*{#1}}}
\newcommand*{\tabref}[1]{\hyperref[#1]{Table~\ref*{#1}}}

\newcommand{\changes}{{\bf\color{red}{Changes}}}
\newcommand{\TODO}{\vskip 4pt {\color{red}\bf TODO}}


\newcommand{\ie}{\emph{i.e.,} }
\newcommand{\cf}{\emph{c.f.} }
\newcommand{\eg}{\emph{e.g.,} }
\newcommand{\etal}{\emph{et~al.}}
\newcommand{\wrt}{w.r.t.~}

\newcommand{\aaron}[1]{{\color{red}\textbf{AT: #1}}}
\newcommand{\derek}[1]{{\color{red}\textbf{DD: #1}}}
\newcommand{\lars}[1]{{\color{red}\textbf{LB: #1}}}
\newcommand{\kasper}[1]{{\color{red}\textbf{KS: #1}}}
\newcommand{\ralf}[1]{{\color{red}\textbf{RJ: #1}}}
\newcommand{\dave}[1]{{\color{red}\textbf{PDS: #1}}}
\newcommand{\hush}[1]{}
\newcommand{\relaxguys}{%
	\let\aaron\hush%
	\let\derek\hush%
	\let\lars\hush%
	\let\kasper\hush%
	\let\ralf\hush%
	\let\dave\hush%
}
