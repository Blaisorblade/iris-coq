% !TEX root = ./appendix.tex
\section{COFE constructions}

\subsection{Next (type-level later)}

Given a COFE $\cofe$, we define $\latert\cofe$ as follows:
\begin{align*}
  \latert\cofe \eqdef{}& \latertinj(\cofe) \\
  \latertinj(x) \nequiv{n} \latertinj(y) \eqdef{}& n = 0 \lor x \nequiv{n-1} y
\end{align*}
$\latert(-)$ is a locally \emph{contractive} functor from $\COFEs$ to $\COFEs$.

\subsection{Uniform Predicates}

Given a CMRA $\monoid$, we define the COFE $\UPred(\monoid)$ of \emph{uniform predicates} over $\monoid$ as follows:
\begin{align*}
  \UPred(\monoid) \eqdef{} \setComp{\pred: \mathbb{N} \times \monoid \to \mProp}{
  \begin{inbox}[c]
    (\All n, x, y. \pred(n, x) \land x \nequiv{n} y \Ra \pred(n, y)) \land {}\\
    (\All n, m, x, y. \pred(n, x) \land x \mincl y \land m \leq n \land y \in \mval_m \Ra \pred(m, y))
  \end{inbox}
}
\end{align*}
where $\mProp$ is the set of meta-level propositions, \eg Coq's \texttt{Prop}.
$\UPred(-)$ is a locally non-expansive functor from $\CMRAs$ to $\COFEs$.

One way to understand this definition is to re-write it a little.
We start by defining the COFE of \emph{step-indexed propositions}: For every step-index, we proposition either holds or does not hold.
\begin{align*}
  \SProp \eqdef{}& \psetdown{\mathbb{N}} \\
    \eqdef{}& \setComp{X \in \pset{\mathbb{N}}}{ \All n, m. n \geq m \Ra n \in X \Ra m \in X } \\
  X \nequiv{n} Y \eqdef{}& \All m \leq n. m \in X \Lra m \in Y
\end{align*}
Notice that with this notion of $\SProp$ is already hidden in the validity predicate $\mval_n$ of a CMRA:
We could equivalently require every CMRA to define $\mval_{-}(-) : \monoid \nfn \SProp$, replacing \ruleref{cmra-valid-ne} and \ruleref{cmra-valid-mono}.

Now we can rewrite $\UPred(\monoid)$ as monotone step-indexed predicates over $\monoid$, where the definition of a ``monotone'' function here is a little funny.
\begin{align*}
  \UPred(\monoid) \cong{}& \monoid \monra \SProp \\
     \eqdef{}& \setComp{\pred: \monoid \nfn \SProp}{\All n, m, x, y. n \in \pred(x) \land x \mincl y \land m \leq n \land y \in \mval_m \Ra m \in \pred(y)}
\end{align*}
The reason we chose the first definition is that it is easier to work with in Coq.

\clearpage
\section{RA and CMRA constructions}

\subsection{Product}
\label{sec:prodm}

Given a family $(M_i)_{i \in I}$ of CMRAs ($I$ finite), we construct a CMRA for the product $\prod_{i \in I} M_i$ by lifting everything pointwise.

Frame-preserving updates on the $M_i$ lift to the product:
\begin{mathpar}
  \inferH{prod-update}
  {\melt \mupd_{M_i} \meltsB}
  {f[i \mapsto \melt] \mupd \setComp{ f[i \mapsto \meltB]}{\meltB \in \meltsB}}
\end{mathpar}

\subsection{Finite partial function}
\label{sec:fpfnm}

Given some countable $K$ and some CMRA $\monoid$, the set of finite partial functions $K \fpfn \monoid$ is equipped with a COFE and CMRA structure by lifting everything pointwise.

We obtain the following frame-preserving updates:
\begin{mathpar}
  \inferH{fpfn-alloc-strong}
  {\text{$G$ infinite} \and \melt \in \mval}
  {\emptyset \mupd \setComp{[\gname \mapsto \melt]}{\gname \in G}}

  \inferH{fpfn-alloc}
  {\melt \in \mval}
  {\emptyset \mupd \setComp{[\gname \mapsto \melt]}{\gname \in K}}

  \inferH{fpfn-update}
  {\melt \mupd \meltsB}
  {f[i \mapsto \melt] \mupd \setComp{ f[i \mapsto \meltB]}{\meltB \in \meltsB}}
\end{mathpar}
$K \fpfn (-)$ is a locally non-expansive functor from $\CMRAs$ to $\CMRAs$.

\subsection{Agreement}

Given some COFE $\cofe$, we define $\agm(\cofe)$ as follows:
\newcommand{\aginjc}{\mathrm{c}} % the "c" field of an agreement element
\newcommand{\aginjV}{\mathrm{V}} % the "V" field of an agreement element
\begin{align*}
  \agm(\cofe) \eqdef{}& \record{\aginjc : \mathbb{N} \to \cofe , \aginjV : \SProp} \\
  & \text{quotiented by} \\
  \melt \equiv \meltB \eqdef{}& \melt.\aginjV = \meltB.\aginjV \land \All n. n \in \melt.\aginjV \Ra \melt.\aginjc(n) \nequiv{n} \meltB.\aginjc(n) \\
  \melt \nequiv{n} \meltB \eqdef{}& (\All m \leq n. m \in \melt.\aginjV \Lra m \in \meltB.\aginjV) \land (\All m \leq n. m \in \melt.\aginjV \Ra \melt.\aginjc(m) \nequiv{m} \meltB.\aginjc(m)) \\
  \mval_n \eqdef{}& \setComp{\melt \in \monoid}{ n \in \melt.\aginjV \land \All m \leq n. \melt.\aginjc(n) \nequiv{m} \melt.\aginjc(m) } \\
  \mcore\melt \eqdef{}& \melt \\
  \melt \mtimes \meltB \eqdef{}& (\melt.\aginjc, \setComp{n}{n \in \melt.\aginjV \land n \in \meltB.\aginjV \land \melt \nequiv{n} \meltB })
\end{align*}
$\agm(-)$ is a locally non-expansive functor from $\COFEs$ to $\CMRAs$.

You can think of the $\aginjc$ as a \emph{chain} of elements of $\cofe$ that has to converge only for $n \in \aginjV$ steps.
The reason we store a chain, rather than a single element, is that $\agm(\cofe)$ needs to be a COFE itself, so we need to be able to give a limit for every chain of $\agm(\cofe)$.
However, given such a chain, we cannot constructively define its limit: Clearly, the $\aginjV$ of the limit is the limit of the $\aginjV$ of the chain.
But what to pick for the actual data, for the element of $\cofe$?
Only if $\aginjV = \mathbb{N}$ we have a chain of $\cofe$ that we can take a limit of; if the $\aginjV$ is smaller, the chain ``cancels'', \ie stops converging as we reach indices $n \notin \aginjV$.
To mitigate this, we apply the usual construction to close a set; we go from elements of $\cofe$ to chains of $\cofe$.

We define an injection $\aginj$ into $\agm(\cofe)$ as follows:
\[ \aginj(x) \eqdef \record{\mathrm c \eqdef \Lam \any. x, \mathrm V \eqdef \mathbb{N}} \]
There are no interesting frame-preserving updates for $\agm(\cofe)$, but we can show the following:
\begin{mathpar}
  \axiomH{ag-val}{\aginj(x) \in \mval_n}

  \axiomH{ag-dup}{\aginj(x) = \aginj(x)\mtimes\aginj(x)}
  
  \axiomH{ag-agree}{\aginj(x) \mtimes \aginj(y) \in \mval_n \Ra x \nequiv{n} y}
\end{mathpar}

\subsection{One-shot}

The purpose of the one-shot CMRA is to lazily initialize the state of a ghost location.
Given some CMRA $\monoid$, we define $\oneshotm(\monoid)$ as follows:
\begin{align*}
  \oneshotm(\monoid) \eqdef{}& \ospending + \osshot(\monoid) + \munit + \bot \\
  \mval_n \eqdef{}& \set{\ospending, \munit} \cup \setComp{\osshot(\melt)}{\melt \in \mval_n}
\\%\end{align*}
%\begin{align*}
  \osshot(\melt) \mtimes \osshot(\meltB) \eqdef{}& \osshot(\melt \mtimes \meltB) \\
  \munit \mtimes \ospending \eqdef{}& \ospending \mtimes \munit \eqdef \ospending \\
  \munit \mtimes \osshot(\melt) \eqdef{}& \osshot(\melt) \mtimes \munit \eqdef \osshot(\melt)
\end{align*}%
The remaining cases of composition go to $\bot$.
\begin{align*}
  \mcore{\ospending} \eqdef{}& \munit & \mcore{\osshot(\melt)} \eqdef{}& \mcore\melt \\
  \mcore{\munit} \eqdef{}& \munit &  \mcore{\bot} \eqdef{}& \bot
\end{align*}
The step-indexed equivalence is inductively defined as follows:
\begin{mathpar}
  \axiom{\ospending \nequiv{n} \ospending}

  \infer{\melt \nequiv{n} \meltB}{\osshot(\melt) \nequiv{n} \osshot(\meltB)}

  \axiom{\munit \nequiv{n} \munit}

  \axiom{\bot \nequiv{n} \bot}
\end{mathpar}
$\oneshotm(-)$ is a locally non-expansive functor from $\CMRAs$ to $\CMRAs$.

We obtain the following frame-preserving updates:
\begin{mathpar}
  \inferH{oneshot-shoot}
  {\melt \in \mval}
  {\ospending \mupd \osshot(\melt)}

  \inferH{oneshot-update}
  {\melt \mupd \meltsB}
  {\osshot(\melt) \mupd \setComp{\osshot(\meltB)}{\meltB \in \meltsB}}
\end{mathpar}

\subsection{Exclusive CMRA}

Given a cofe $\cofe$, we define a CMRA $\exm(\cofe)$ such that at most one $x \in \cofe$ can be owned:
\begin{align*}
  \exm(\cofe) \eqdef{}& \exinj(\cofe) + \munit + \bot \\
  \mval_n \eqdef{}& \setComp{\melt\in\exm(\cofe)}{\melt \neq \bot} \\
  \munit \mtimes \exinj(x) \eqdef{}& \exinj(x) \mtimes \munit \eqdef \exinj(x)
\end{align*}
The remaining cases of composition go to $\bot$.
\begin{align*}
  \mcore{\exinj(x)} \eqdef{}& \munit & \mcore{\munit} \eqdef{}& \munit &
  \mcore{\bot} \eqdef{}& \bot
\end{align*}
The step-indexed equivalence is inductively defined as follows:
\begin{mathpar}
  \infer{x \nequiv{n} y}{\exinj(x) \nequiv{n} \exinj(y)}

  \axiom{\munit \nequiv{n} \munit}

  \axiom{\bot \nequiv{n} \bot}
\end{mathpar}
$\exm(-)$ is a locally non-expansive functor from $\COFEs$ to $\CMRAs$.

We obtain the following frame-preserving update:
\begin{mathpar}
  \inferH{ex-update}{}
  {\exinj(x) \mupd \exinj(y)}
\end{mathpar}



%TODO: These need syncing with Coq
% \subsection{Finite Powerset Monoid}

% Given an infinite set $X$, we define a monoid $\textmon{PowFin}$ with carrier $\mathcal{P}^{\textrm{fin}}(X)$ as follows:
% \[
% \melt \cdot \meltB \;\eqdef\; \melt \cup \meltB \quad \mbox{if } \melt \cap \meltB = \emptyset
% \]

% We obtain:
% \begin{mathpar}
% 	\inferH{PowFinUpd}{}
% 		{\emptyset \mupd \{ \{x\} \mid x \in X  \}}
% \end{mathpar}

% \begin{proof}[Proof of \ruleref{PowFinUpd}]
% 	Assume some frame $\melt_\f \sep \emptyset$. Since $\melt_\f$ is finite and $X$ is infinite, there exists an $x \notin \melt_\f$.
% 	Pick that for the result.
% \end{proof}

% The powerset monoids is cancellative.
% \begin{proof}[Proof of cancellativity]
% 	Let $\melt_\f \mtimes \melt = \melt_\f \mtimes \meltB \neq \mzero$.
% 	So we have $\melt_\f \sep \melt$ and $\melt_\f \sep \meltB$, and we have to show $\melt = \meltB$.
% 	Assume $x \in \melt$. Hence $x \in \melt_\f \mtimes \melt$ and thus $x \in \melt_\f \mtimes \meltB$.
% 	By disjointness, $x \notin \melt_\f$ and hence $x \in meltB$.
% 	The other direction works the same way.
% \end{proof}


% \subsection{Fractional monoid}
% \label{sec:fracm}

% Given a monoid $M$, we define a monoid representing fractional ownership of some piece $\melt \in M$.
% The idea is to preserve all the frame-preserving update that $M$ could have, while additionally being able to do \emph{any} update if we own the full state (as determined by the fraction being $1$).
% Let $\fracm{M}$ be the monoid with carrier $(((0, 1] \cap \mathbb{Q}) \times M) \uplus \{\munit\}$ and multiplication
% \begin{align*}
%  (q, a) \mtimes (q', a') &\eqdef (q + q', a \mtimes a') \qquad \mbox{if $q+q'\le 1$} \\
%  (q, a) \mtimes \munit &\eqdef (q,a) \\
%  \munit \mtimes (q,a) &\eqdef (q,a).
% \end{align*}

% We get the following frame-preserving update.
% \begin{mathpar}
% 	\inferH{FracUpdFull}
% 		{a, b \in M}
% 		{(1, a) \mupd (1, b)}
%   \and\inferH{FracUpdLocal}
% 	  {a \mupd_M B}
% 	  {(q, a) \mupd \{q\} \times B}
% \end{mathpar}

% \begin{proof}[Proof of \ruleref{FracUpdFull}]
% Assume some $f \sep (1, a)$. This can only be $f = \munit$, so showing $f \sep (1, b)$ is trivial.
% \end{proof}

% \begin{proof}[Proof of \ruleref{FracUpdLocal}]
% 	Assume some $f \sep (q, a)$. If $f = \munit$, then $f \sep (q, b)$ is trivial for any $b \in B$. Just pick the one we obtain by choosing $\munit_M$ as the frame for $a$.
	
% 	In the interesting case, we have $f = (q_\f, a_\f)$.
% 	Obtain $b$ such that $b \in B \land b \sep a_\f$.
% 	Then $(q, b) \sep f$, and we are done.
% \end{proof}

% $\fracm{M}$ is cancellative if $M$ is cancellative.
% \begin{proof}[Proof of cancellativitiy]
% If $\melt_\f = \munit$, we are trivially done.
% So let $\melt_\f = (q_\f, \melt_\f')$.
% If $\melt = \munit$, then $\meltB = \munit$ as otherwise the fractions could not match up.
% Again, we are trivially done.
% Similar so for $\meltB = \munit$.
% So let $\melt = (q_a, \melt')$ and $\meltB = (q_b, \meltB')$.
% We have $(q_\f + q_a, \melt_\f' \mtimes \melt') = (q_\f + q_b, \melt_\f' \mtimes \meltB')$.
% We have to show $q_a = q_b$ and $\melt' = \meltB'$.
% The first is trivial, the second follows from cancellativitiy of $M$.
% \end{proof}


% %\subsection{Disposable monoid}
% %
% %Given a monoid $M$, we construct a monoid where, having full ownership of an element $\melt$ of $M$, one can throw it away, transitioning to a dead element.
% %Let \dispm{M} be the monoid with carrier $\mcarp{M} \uplus \{ \disposed \}$ and multiplication
% %% The previous unit must remain the unit of the new monoid, as is is always duplicable and hence we could not transition to \disposed if it were not composable with \disposed
% %\begin{align*}
% %  \melt \mtimes \meltB &\eqdef \melt \mtimes_M \meltB & \IF \melt \sep[M] \meltB \\
% %  \disposed \mtimes \disposed &\eqdef \disposed \\
% %  \munit_M \mtimes \disposed &\eqdef \disposed \mtimes \munit_M \eqdef \disposed
% %\end{align*}
% %The unit is the same as in $M$.
% %
% %The frame-preserving updates are
% %\begin{mathpar}
% % \inferH{DispUpd}
% %   {a \in \mcarp{M} \setminus \{\munit_M\} \and a \mupd_M B}
% %   {a \mupd B}
% % \and
% % \inferH{Dispose}
% %  {a \in \mcarp{M} \setminus \{\munit_M\} \and \All b \in \mcarp{M}. a \sep b \Ra b = \munit_M}
% %  {a \mupd \disposed}
% %\end{mathpar}
% %
% %\begin{proof}[Proof of \ruleref{DispUpd}]
% %Assume a frame $f$. If $f = \disposed$, then $a = \munit_M$, which is a contradiction.
% %Thus $f \in \mcarp{M}$ and we can use $a \mupd_M B$.
% %\end{proof}
% %
% %\begin{proof}[Proof of \ruleref{Dispose}]
% %The second premiss says that $a$ has no non-trivial frame in $M$. To show the update, assume a frame $f$ in $\dispm{M}$. Like above, we get $f \in \mcarp{M}$, and thus $f = \munit_M$. But $\disposed \sep \munit_M$ is trivial, so we are done.
% %\end{proof}

% \subsection{Authoritative monoid}\label{sec:auth}

% Given a monoid $M$, we construct a monoid modeling someone owning an \emph{authoritative} element $x$ of $M$, and others potentially owning fragments $\melt \le_M x$ of $x$.
% (If $M$ is an exclusive monoid, the construction is very similar to a half-ownership monoid with two asymmetric halves.)
% Let $\auth{M}$ be the monoid with carrier
% \[
% 	\setComp{ (x, \melt) }{ x \in \mcarp{\exm{\mcarp{M}}} \land \melt \in \mcarp{M} \land (x = \munit_{\exm{\mcarp{M}}} \lor \melt \leq_M x) }
% \]
% and multiplication
% \[
% (x, \melt) \mtimes (y, \meltB) \eqdef
%      (x \mtimes y, \melt \mtimes \meltB) \quad \mbox{if } x \sep y \land \melt \sep \meltB \land (x \mtimes y = \munit_{\exm{\mcarp{M}}} \lor \melt \mtimes \meltB \leq_M x \mtimes y)
% \]
% Note that $(\munit_{\exm{\mcarp{M}}}, \munit_M)$ is the unit and asserts no ownership whatsoever, but $(\munit_{M}, \munit_M)$ asserts that the authoritative element is $\munit_M$.

% Let $x, \melt \in \mcarp M$.
% We write $\authfull x$ for full ownership $(x, \munit_M):\auth{M}$ and $\authfrag \melt$ for fragmental ownership $(\munit_{\exm{\mcarp{M}}}, \melt)$ and $\authfull x , \authfrag \melt$ for combined ownership $(x, \melt)$.
% If $x$ or $a$ is $\mzero_{M}$, then the sugar denotes $\mzero_{\auth{M}}$.

% \ralf{This needs syncing with the Coq development.}
% The frame-preserving update involves a rather unwieldy side-condition:
% \begin{mathpar}
% 	\inferH{AuthUpd}{
% 		\All\melt_\f\in\mcar{\monoid}. \melt\sep\meltB \land \melt\mtimes\melt_\f \le \meltB\mtimes\melt_\f \Ra \melt'\mtimes\melt_\f \le \melt'\mtimes\meltB \and
% 		\melt' \sep \meltB
% 	}{
% 		\authfull \melt \mtimes \meltB, \authfrag \melt \mupd \authfull \melt' \mtimes \meltB, \authfrag \melt'
% 	}
% \end{mathpar}
% We therefore derive two special cases.

% \paragraph{Local frame-preserving updates.}

% \newcommand\authupd{f}%
% Following~\cite{scsl}, we say that $\authupd: \mcar{M} \ra \mcar{M}$ is \emph{local} if
% \[
% 	\All a, b \in \mcar{M}. a \sep b \land \authupd(a) \neq \mzero \Ra \authupd(a \mtimes b) = \authupd(a) \mtimes b
% \]
% Then,
% \begin{mathpar}
% 	\inferH{AuthUpdLocal}
% 	{\text{$\authupd$ local} \and \authupd(\melt)\sep\meltB}
% 	{\authfull \melt \mtimes \meltB, \authfrag \melt \mupd \authfull \authupd(\melt) \mtimes \meltB, \authfrag \authupd(\melt)}
% \end{mathpar}

% \paragraph{Frame-preserving updates on cancellative monoids.}

% Frame-preserving updates are also possible if we assume $M$ cancellative:
% \begin{mathpar}
%  \inferH{AuthUpdCancel}
%   {\text{$M$ cancellative} \and \melt'\sep\meltB}
%   {\authfull \melt \mtimes \meltB, \authfrag \melt \mupd \authfull \melt' \mtimes \meltB, \authfrag \melt'}
% \end{mathpar}

% \subsection{Fractional heap monoid}
% \label{sec:fheapm}

% By combining the fractional, finite partial function, and authoritative monoids, we construct two flavors of heaps with fractional permissions and mention their important frame-preserving updates.
% Hereinafter, we assume the set $\textdom{Val}$ of values is countable.

% Given a set $Y$, define $\FHeap(Y) \eqdef \textdom{Val} \fpfn \fracm(Y)$ representing a fractional heap with codomain $Y$.
% From \S\S\ref{sec:fracm} and~\ref{sec:fpfunm} we obtain the following frame-preserving updates as well as the fact that $\FHeap(Y)$ is cancellative.
% \begin{mathpar}
% 	\axiomH{FHeapUpd}{h[x \mapsto (1, y)] \mupd h[x \mapsto (1, y')]} \and
% 	\axiomH{FHeapAlloc}{h \mupd \{\, h[x \mapsto (1, y)] \mid x \in \textdom{Val} \,\}}
% \end{mathpar}
% We will write $qh$ with $h : \textsort{Val} \fpfn Y$ for the function in $\FHeap(Y)$ mapping every $x \in \dom(h)$ to $(q, h(x))$, and everything else to $\munit$.

% Define $\AFHeap(Y) \eqdef \auth{\FHeap(Y)}$ representing an authoritative fractional heap with codomain $Y$.
% We easily obtain the following frame-preserving updates.
% \begin{mathpar}
% 	\axiomH{AFHeapUpd}{
% 		(\authfull h[x \mapsto (1, y)], \authfrag [x \mapsto (1, y)]) \mupd (\authfull h[x \mapsto (1, y')], \authfrag [x \mapsto (1, y')])
% 	}
% 	\and
% 	\inferH{AFHeapAdd}{
% 		x \notin \dom(h)
% 	}{
% 		\authfull h \mupd (\authfull h[x \mapsto (q, y)], \authfrag [x \mapsto (q, y)])
% 	}
% 	\and
% 	\axiomH{AFHeapRemove}{
% 		(\authfull h[x \mapsto (q, y)], \authfrag [x \mapsto (q, y)]) \mupd \authfull h
% 	}
% \end{mathpar}

\subsection{STS with tokens}
\label{sec:stsmon}

Given a state-transition system~(STS, \ie a directed graph) $(\STSS, {\stsstep} \subseteq \STSS \times \STSS)$, a set of tokens $\STST$, and a labeling $\STSL: \STSS \ra \wp(\STST)$ of \emph{protocol-owned} tokens for each state, we construct a monoid modeling an authoritative current state and permitting transitions given a \emph{bound} on the current state and a set of \emph{locally-owned} tokens.

The construction follows the idea of STSs as described in CaReSL \cite{caresl}.
We first lift the transition relation to $\STSS \times \wp(\STST)$ (implementing a \emph{law of token conservation}) and define a stepping relation for the \emph{frame} of a given token set:
\begin{align*}
 (s, T) \stsstep (s', T') \eqdef{}& s \stsstep s' \land \STSL(s) \uplus T = \STSL(s') \uplus T' \\
 s \stsfstep{T} s' \eqdef{}& \Exists T_1, T_2. T_1 \sep \STSL(s) \cup T \l+and (s, T_1) \stsstep (s', T_2)
\end{align*}

We further define \emph{closed} sets of states (given a particular set of tokens) as well as the \emph{closure} of a set:
\begin{align*}
\STSclsd(S, T) \eqdef{}& \All s \in S. \STSL(s) \sep T \land \All s'. s \stsfstep{T} s' \Ra s' \in S \\
\upclose(S, T) \eqdef{}& \setComp{ s' \in \STSS}{\Exists s \in S. s \stsftrans{T} s' }
\end{align*}

The STS RA is defined as follows
\begin{align*}
  \monoid \eqdef{}& \setComp{\STSauth((s, T) \in \STSS \times \wp(\STST))}{\STSL(s) \sep T} +{}\\& \setComp{\STSfrag((S, T) \in \wp(\STSS) \times \wp(\STST))}{\STSclsd(S, T) \land S \neq \emptyset} + \bot \\
  \STSfrag(S_1, T_1) \mtimes \STSfrag(S_2, T_2) \eqdef{}& \STSfrag(S_1 \cap S_2, T_1 \cup T_2) \qquad\qquad\qquad \text{if $T_1 \sep T_2$ and $S_1 \cap S_2 \neq \emptyset$} \\
  \STSfrag(S, T) \mtimes \STSauth(s, T') \eqdef{}& \STSauth(s, T') \mtimes \STSfrag(S, T) \eqdef \STSauth(s, T \cup T') \qquad \text{if $T \sep T'$ and $s \in S$} \\
  \mcore{\STSfrag(S, T)} \eqdef{}& \STSfrag(\upclose(S, \emptyset), \emptyset) \\
  \mcore{\STSauth(s, T)} \eqdef{}& \STSfrag(\upclose(\set{s}, \emptyset), \emptyset)
\end{align*}
The remaining cases are all $\bot$.

We will need the following frame-preserving update:
\begin{mathpar}
  \inferH{sts-step}{(s, T) \ststrans (s', T')}
  {\STSauth(s, T) \mupd \STSauth(s', T')}

  \inferH{sts-weaken}
  {\STSclsd(S_2, T_2) \and S_1 \subseteq S_2 \and T_2 \subseteq T_1}
  {\STSfrag(S_1, T_1) \mupd \STSfrag(S_2, T_2)}
\end{mathpar}

\paragraph{The core is not a homomorphism.}
The core of the STS construction is only satisfying the RA axioms because we are \emph{not} demanding the core to be a homomorphism---all we demand is for the core to be monotone with respect the \ruleref{ra-incl}.

In other words, the following does \emph{not} hold for the STS core as defined above:
\[ \mcore\melt \mtimes \mcore\meltB = \mcore{\melt\mtimes\meltB} \]

To see why, consider the following STS:
\newcommand\st{\textlog{s}}
\newcommand\tok{\textmon{t}}
\begin{center}
  \begin{tikzpicture}[sts]
    \node at (0,0)   (s1) {$\st_1$};
    \node at (3,0)  (s2) {$\st_2$};
    \node at (9,0) (s3) {$\st_3$};
    \node at (6,0)  (s4) {$\st_4$\\$[\tok_1, \tok_2]$};
    
    \path[sts_arrows] (s2) edge  (s4);
    \path[sts_arrows] (s3) edge  (s4);
  \end{tikzpicture}
\end{center}
Now consider the following two elements of the STS RA:
\[ \melt \eqdef \STSfrag(\set{\st_1,\st_2}, \set{\tok_1}) \qquad\qquad
  \meltB \eqdef \STSfrag(\set{\st_1,\st_3}, \set{\tok_2}) \]

We have:
\begin{mathpar}
  {\melt\mtimes\meltB = \STSfrag(\set{\st_1}, \set{\tok_1, \tok_2})}

  {\mcore\melt = \STSfrag(\set{\st_1, \st_2, \st_4}, \emptyset)}

  {\mcore\meltB = \STSfrag(\set{\st_1, \st_3, \st_4}, \emptyset)}

  {\mcore\melt \mtimes \mcore\meltB = \STSfrag(\set{\st_1, \st_4}, \emptyset) \neq
    \mcore{\melt \mtimes \meltB} = \STSfrag(\set{\st_1}, \emptyset)}
\end{mathpar}

%%% Local Variables: 
%%% mode: latex
%%% TeX-master: "iris"
%%% End: 
